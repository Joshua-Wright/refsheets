% # (c) Copyright 2015 Josh Wright
\documentclass{article}
\usepackage{mathrsfs,amsmath,amsthm,latexsym,paralist}
\usepackage{mathtools} 
\usepackage{bm}
\usepackage{calc}      % for dimension arithmetic
\usepackage{amssymb}   % for \varnothing, \therefore
\usepackage{centernot} % for \centernot
\usepackage{geometry}  % for margins
\usepackage{outlines}  % for outline
% \usepackage{paralist}  % for compactitem, compactenum
\usepackage{multicol}
\usepackage{hyperref}
\hypersetup{
  pdftitle={MATH 308 Reference Sheet},
  pdfauthor={Josh Wright},
  pdfsubject={MATH 308},
  bookmarksnumbered=true,
  bookmarksopen=true,
  bookmarksopenlevel=1,
  colorlinks=true,
  pdfstartview=Fit,
  pdfpagemode=UseOutlines,
  colorlinks=true,
  linkcolor=blue,
  filecolor=magenta,      
  urlcolor=cyan,
}
%%%%%%%%%%%%%%%%%%%%%%%%%%%%%%%%%%%%%%
%% paper size, orientation, margins %%
%%%%%%%%%%%%%%%%%%%%%%%%%%%%%%%%%%%%%%
% allow hboxes to be overfull
\hfuzz=10in
% good for on screen-only viewing
\def \columncount {3}
\newlength{\papersize}
\setlength{\papersize}{16cm}
\geometry{paperheight=1.7777\papersize, paperwidth=\papersize, landscape, margin=0.25in}
% good for printing
% \def \columncount {2}
% \geometry{letterpaper, portrait, margin=0.25in}


% makes second-level itemize bullets instead of dashes
% \renewcommand\labelitemi{\cdot}
\renewcommand\labelitemi{\tiny$\bullet$}
\renewcommand\labelitemii{\labelitemi}

%%%%%%%%%%%%%%%%%%%%
%% helpful macros %%
%%%%%%%%%%%%%%%%%%%%
\newcommand{\p}{\partial}
\newcommand{\ang}[1]{\left\langle #1 \right\rangle}
\newcommand{\ceil}[1]{\left\lceil #1 \right\rceil}
\newcommand{\floor}[1]{\left\lfloor #1 \right\rfloor}
\newcommand{\prt}[2]{\frac{\partial#1}{\partial#2}}
\newcommand{\grad}{\nabla}


\begin{document}
\allowdisplaybreaks
% \begin{center}
\large
\noindent
\textbf{MATH 308 Reference Sheet} \hfill Last Updated: \today \hfill \textcopyright \space Josh Wright 2016
\\ Paul's Online Math Notes: \url{http://tutorial.math.lamar.edu/Classes/DE/DE.aspx}
\\ Latex Symbols: \url{http://oeis.org/wiki/List_of_LaTeX_mathematical_symbols}
% \end{center}
% asterisk makes multicols finish one column before going onto the next
\begin{multicols*}{\columncount}
\begin{outline}[compactitem]



%%%%%%%%%%%%%%%%%%%%
%% spacing config %%
%%%%%%%%%%%%%%%%%%%%
% just in case I need even more space
\newcommand{\upspace}{\vspace{0px}\linespread{0}}
% section titles
\newcommand{\zzz}[1]{\noindent\0\noindent {\textbf{#1:}}}
% makes second-level itemize bullets instead of dashes
\renewcommand\labelitemii{\labelitemi}
% redefine the sub-headings to inject our space-saver
\let\oldOne\1\let\oldTwo\2\let\oldThree\3\let\oldFour\4
\renewcommand{\1}{\upspace \oldOne  }
\renewcommand{\2}{\upspace \oldTwo  }
\renewcommand{\3}{\upspace \oldThree}
\renewcommand{\4}{\upspace \oldFour }






\noindent
\zzz{First Order Linear: Interval of Validity}
  \1 find the $x$ values for which the differential equation is undefined or discontinuous
  \1 split up the number line into intervals among those points
  \1 pick the interval that contains your initial condition
\zzz{First Order Linear: Separable}
  \1 coerce into form $f(y) dy = g(x) dx$
  \\ (use $y' = \frac{dy}{dx}$)
  \1 integrate both sides
  \1 solve for y

\zzz{First Order Linear: Integrating Factor}
  \1 form $y' + p(x)y = g(x)$
  \1 $\mu(x) = e^{\int_0^x p(s) ds}$
  \1 multiply both sides by $\mu(x)$
  \1 now it's $\frac{d}{dx}(\mu(x)y) = \mu(x)g(x)$
  \\ $\mu(x)y = \int \mu(x)g(x) dx$

\zzz{First Order Linear: Exact Equations}
  \1 form: $M(x,y) + N(x,y)\frac{dy}{dx} = 0$
  \1 $\Psi_x = M, \Psi_y = N$, find $\Psi(x,y)=\ldots$
  \1 Exact if $M_y == N_x$
  \\ $\Psi(x,y) = \int M dx + h(y)$
  \\ $\Psi(x,y) = \int N dx + g(x)$
\zzz{Inexact $\rightarrow$ Exact}
  \1 $\frac{d\mu(x)}{dx} = \frac{M_y-N_x}{N}\mu(x)$ only applies if $y$ drops out
  \\ (similar with $\mu(y)$)

\zzz{Homogeneous}
  % \1 $ay'' + by' + cy = 0$
  \1 $y'' + p(t)y' + q(t)y = 0$
  \1 Characteristic Equation: $ar^2 + br + c = 0$
  \1 General Solution: 
  \\ $y(t) = c_1 e^{r_1t} + c_2 e^{r_2t}$
  % \\ $y_{1,2}(t) = e^{r_{1,2} t}$
  \1 if $r_1 = r_2$ then $y_2 = te^{rt}$
  \1 Complex roots: 
    \2 $r = \lambda \pm \mu i$
    \2 $y_{1,2}(t) = e^{(\lambda \pm \mu i) t}$
    \2 $u(t) = e^{\lambda t}\cos(\mu t)$
    \2 $v(t) = e^{\lambda t}\sin(\mu t)$
    \2 $y(t) = c_1 e^{\lambda t}\cos(\mu t) + c_2 e^{\lambda t}\sin(\mu t)$
  \1 $\lambda$ ends up as negative because of $\frac{-b}{2a}$ in the quadratic formula
  \1 $\mu$ is positive (the imaginary part of the solution)

\zzz{Wronskian}
  \1 $y'' + p(t)y' + q(t)y = 0$
  \1 $W(y_1,y_2)(t) = y_1(t)y_2'(t) - y_2(t)y_1'(t) = Ce^{-\int_0^t p(s) ds}$

\zzz{Non-Homogeneous: undetermined coefficients}
  \1 $y'' + p(t)y' + q(t)y = g(t)$ 
  \1 first replace $g(t)$ with $0$ and find $y_1(t), y_2(t)$ (the complimentary solution $y_c(t)$)
  \1 then find $y_p$: $y(t) = y_1(t) + y_2(t) + y_p(t)$
  \1 guess a $y_p$ (usually a combination from $Ce^t$, $C_1\sin(C_2t)$, $C_1\cos(C_2t)$, or polynomial)
  \1 substitute $y(t)\rightarrow y_p$ in the original equation and
  \1 solve for unknown coefficients
  \1 final solution is $y(t) = y_c(t) + y_p(t)$

\zzz{Variation of Parameters}
  \1 $y'' + p(t)y' + q(t)y = g(t)$ 
  \\ First solve the complimentary homogeneous equation
  \1 $y = u_1(t)y_1(t) + u_2(t)y_2(t)$

\zzz{Damped Mass on a  Spring}
  \1 $mx'' + \gamma x' + kx = f(t)$
  \1 equilibrium position: $x=0$
  \1 $x=$ position; initial $x(0)=h$
  \1 velocity: $v=x'$; initial $x'(0)=v_0$
  \1 acceleration: $a=x''$
  \1 $k=$ spring constant
  \1 $m=$ mass
  \1 $\omega=$ period $=\sqrt{\frac{k}{m}}$ (when undamped)
  \1 friction determined by $\gamma$
    \2 (situation dependent)
    \2 has units force per velocity
  \1 external force per time: $f(t)$
  \1 if $f(t)=0$ and $\gamma=0$ (no friction)
    \2 $x(t)=h\cos(\omega t) + \frac{v_0}{\omega}\sin(\omega t)$
    \2 $x(t) = A\cos(\omega t + \sigma)$
    \\ $A = \sqrt{h^2 + (\frac{v_0}{\omega})^2}$
    \\ $\sigma = -\tan^{-1}\left(\frac{hv_0}{\omega}\right)$
  \1 damping type depends on how $\gamma^2$ relates to $4km$
  \1 if $f(t)=0$ and overdamped: \hfill $\gamma^2 > 4km$
    \2 happens when $\gamma^2 > 4km$
      \3 Therefore $r_1,r_2 > 0$
    \2 $x(t) = C_1 e^{r_1 t} + C_2 e^{r_2 t}$
    \2 no oscillation, $x(t) \rightarrow 0$ as $t \rightarrow \infty$
  \1 if $f(t)=0$ and underdamped: \hfill $\gamma^2 < 4km$
    \2 happens when $\gamma^2 < 4km$
      \3 $r_1,r_2$ are complex
    \2 oscillates forever, amplitude approaching 0
    \2 $x(t) = e^{-\frac{\gamma}{2m}t}\left( C_1\cos(\omega t) + C_2 \sin(\omega t)  \right) $
    \2 $x(t) = Ae^{-\frac{\gamma}{2m}t}\cos(\omega t + \sigma)$
      \3 different $A,\sigma$ than no friction scenario
  \1 if $f(t)=0$ and critically damped: \hfill $\gamma^2 = 4km$
    \2 happens when $\gamma^2 = 4km$
    \2 returns to $x=0$ as quickly as possible without oscillating
    (exponentially decays toward $x=0$ as $t\rightarrow\infty$)
    \2 $x(t) = e^{-\frac{\gamma}{2m}t} (C_1 + C_2t)$
      \3 only one $r$ solution
      \3 $x(0)=h=C_1$
    \2 if $C_1,C_2$ have the same sign:
    \\ $x(t)$ is always on the same side of the $x$-axis
    \2 if $C_1,C_2$ have opposite sign:
    \\ $x(t)$ must cross the the $x$-axis exactly once
  \1 if $f(t) = F_0$
    \2 constant external force
    \2 $x_p = A$
    \2 $x_c = $ complimentary solution for $f(t)=0$
    \2 $x(t) = x_c + x_p$
    \2 $x(t) \rightarrow A$ as $t\rightarrow\infty$
  \1 if $f(t) = F_0 \cos(\omega_2 t)$
    \2 external periodic force
    \2 $x_p = A\cos(\omega_2 t) + B\sin(\omega_2 t)$
    \2 oscillates with constant period $T=\frac{2k}{\omega_2}$ as $t\rightarrow\infty$ (because original oscillation dies out)
  \1 if $x_p$ solves $x_c$:
    \2 instead use $x_p = t ( A\cos(\omega_2 t) + B\sin(\omega_2 t) )$
    \2 means you have \textbf{resonance}; 
    means that the force perfectly adds to the existing kinetic energy; 
    if $\gamma$ is small, amplitude can grow large

\zzz{Electrical Vibrations: Series Circuit}
  \1 equation: $LQ'' + RQ' + \frac{1}{C}Q = E(t)$
  \1 $E(t)$: impressed (input) voltage
  \1 derivative therefore in terms of current: $LI'' + RI' + \frac{1}{C}I = E'(t)$

\zzz{Series Solutions}
  \1 For equation with order $n$, you need initial conditions for $y',y'',\ldots,y^{(n-1)}$
  \1 you can then find initial $y^{(n)}$ by solving the equation for it with the initial conditions
  \1 And you can then repeatedly differentiate the equation with respect to $x$ to find lots of initial condition derivatives
  \1 then find the formula for the $n^{th}$ derivative and plug it into the taylor series formula
  \1 \textbf{Taylor Series Formula:} $\sum_{n=0}^{\infty}\frac{f^{(n)}(a)}{n!}(x-a)^n$
  \1 Maclaurin series is just the taylor series with $a=0$ (centered at 0)
  \1\begin{tabular}{r c l}
    $f(x)     $ & $=$ & $\sum a_n x^n$                    \\
    $f'(x)    $ & $=$ & $\sum_{n=1} (n) a_n x^{n-1}$      \\
    $f'(x)    $ & $=$ & $\sum_{n=0} (n+1) a_{n+1} x^{n}$  \\
    $f''(x)   $ & $=$ & $\sum_{n=2} n(n-1)a_{n}x^{n-2}$   \\
    $f''(x)   $ & $=$ & $\sum_{n=0} (n+2)(n+1)a_{n+2}x^n$ \\
    $f(x)+g(x)$ & $=$ & $\sum (a_n+b_n)x^n$               \\
    \end{tabular}
  \1 solve using: $y''+Py'+Qy=0 \rightarrow \sum[(n+2)(n+1)a_{n+2} + P(n+1)a_{n+1} + Qa_n]x^n=0$

\zzz{Matrix Solutions}
  \1 works for any $n^{th}$ order equation $y^{(n)} + P_{n-1}y^{(n-1)} \ldots P_1y' + P_0$

\zzz{Laplace Transform}
  \1 $\mathscr{L}[f(t)](s)=F(s)=\int_{0}^{\infty} e^{-st} f(t) dt$
  \1 convention is that an uppercase function in $s$ is the Laplace transform of a lowercase function in $t$
  \1 generally use a lookup table
  \1 $n$th derivative: 
    $\mathscr{L}[f^{(n)}(t)] = s^nF(s) - s^{n-1}y(0) - s^{n-2}y'(0) \ldots - y^{(n-1)}(0)$
  \1 common Laplace transforms:
    \\\begin{tabular}{l l l}
      $f(t)$: & $F(s)$: & validity: \\ \hline
      $f(t) + g(t)$ & $F(s) + G(s)$ & all $s$ \\
      $C*f(t)$ & $C*F(s)$ & all $s$ \\
      $t*f(t)$ & $-F'(s)$ & $s>0$? \\
      $e^{at}$ & $\frac{1}{s-a}$ & $s > a$ \\
      $\sin(a t)$ & $\frac{a}{s^2+a^2}$ & $s > 0$ \\
      $\cos(a t)$ & $\frac{s}{s^2+a^2}$ & $s > 0$ \\
      $1$ & $\frac{1}{s}$ & $s > 0$ \\
      $t$ & $\frac{1}{s^2}$ & $s > 0$ \\
      $t^n$ & $\frac{n!}{s^{n+1}}$ & $s > 0$ \\
      $f(t)e^{ct}$ & $F(s-c)$ & $s > c$? \\
      $u(t-a)g(t-a)$ & $G(s)e^{-as}$ & $s > 0$? \\
      $\delta(t-a)$ & $e^{-as}$ &  \\
      % $$ & $$ & $$ \\
    \end{tabular}
    \\\begin{tabular}{l l}
    $\mathscr{L}[f(t)](s)   $&$=Y(s)$ \\
    $\mathscr{L}[f'(t)](s)  $&$=sY(s)-y(0)$ \\
    $\mathscr{L}[f''(t)](s) $&$=s^2Y(s)-sy(0)-y'(0)$ \\
    \end{tabular}
  \1 generally you don't do Laplace transforms by hand, you use a lookup table (same for inverse)

\zzz{Laplace - Unit Step Function}
  \1 $u(t)=\{_{0 : t < 0}^{1 : t > 0}$
  \1 $\mathscr{L}[u(t-a)]=\frac{1}{s}e^{-as}$
  \1 $\mathscr{L}[u(t-a)g(t-a)]=G(s)e^{-as}$

\zzz{Laplace - Convolution}
  \1 $(f*g)(t)=\int_{0}^{t}f(t-T)g(T)dt$
  \1 $(f*g)(t)= (g*f)(t) $
  \1 $\mathscr{L}[(f*g)(t)]=F(s)G(s)$

\zzz{Laplace - Derivative}
  \1 $\mathscr{L}[tf(t)]=-F'(s)$
  \1 $\mathscr{L}[t^2f(t)]=F''(s)$
  \1 $\mathscr{L}[t^nf(t)]=(-1)^nF^{(n)}(s)$




\zzz{\Large Final Exam Stuff}

\zzz{Linear Algebra}
  \1 matrix manipulations:
    \2 for when you've got a square matrix (right hand side) and column matrix (left hand side)
    \2 multiply a row by a constant
    \2 add rows together and store in existing row
    \2 subtract rows, store in existing row
  \1 equation as matrix:
    \2 a row is all 0's: solution is not unique
    \2 all 0's row except equal to non-zero: equations are inconsistent, no (real) solution exists
    \2 



\end{outline}
\end{multicols*}
\end{document}

% Josh_Wright_Resume.tex
% (c) Copyright 2015 Josh Wright
\documentclass[12pt]{article}
\usepackage{verbatim}
% \usepackage{syntonly}
\usepackage{ragged2e}
\usepackage{geometry}
\usepackage{enumitem} % for longenum
\usepackage{setspace}
\usepackage{hyperref}
\usepackage{tabularx}
\usepackage{outlines} % for outline
\usepackage{paralist} % for compactitem (compact itemize)
\usepackage{multicol} % for multicolumn layout
\geometry{letterpaper, margin=0.5in, top=0.3in}
% \geometry{letterpaper, margin=0.5in, top=0.35in, left=1.5in}

\begin{document}
% \linespread{0.5}
\begin{center}
Math 311 Reference Sheet
\hfill \textcopyright{} Josh Wright 2015 \hfill
Last Updated: \today
\end{center}

%%%%%%%%%%%%%%%%%%
%% main section %%
%%%%%%%%%%%%%%%%%%
\begin{multicols*}{2}
\begin{flushleft}
\newlist{longenum}{itemize}{5}
\setlist[longenum,1]{nosep,leftmargin=0.4cm,labelwidth=0px,align=left,label=$\bullet$}
\setlist[longenum,2]{nosep,leftmargin=0.4cm,labelwidth=0px,align=left,label=$\ast$}
\setlist[longenum,3]{nosep,leftmargin=0.4cm,labelwidth=0px,align=left,label=-}
\setlist[longenum,4]{nosep,leftmargin=0.4cm,labelwidth=0px,align=left,label=>}
\setlist[longenum,5]{nosep,leftmargin=0.4cm,labelwidth=0px,align=left,label=@}
% \begin{outline}[compactitem]
\begin{outline}[longenum]

%%%%%%%%%%%%%%%%%%%%
%% spacing config %%
%%%%%%%%%%%%%%%%%%%%
% just in case I need even more space
\newlength{\upspacelength}
\setlength{\upspacelength}{0px}
\newcommand{\upspace}{\vspace{\upspacelength}}
% section titles
\newcommand{\zzz}[1]{\upspace\0 \textbf{#1} }
% \newcommand{\zzz}[1]{\0 \hspace{-1.25in} \textbf{#1} \vspace{-10px} }
% makes second-level itemize bullets instead of dashes
% \renewcommand\labelitemii{\labelitemi}
% redefine the sub-headings to inject our space-saver
\let\oldOne\1\let\oldTwo\2\let\oldThree\3\let\oldFour\4
\renewcommand{\1}{\upspace{}\oldOne{}}
\renewcommand{\2}{\upspace{}\oldTwo{}}
\renewcommand{\3}{\upspace{}\oldThree{}}
\renewcommand{\4}{\upspace{}\oldFour{}}
%%%%%%%%%%%%%%%%%%%%%%%%%%%%%%%%%%%%%%%%%%%%%%%%%%%%%%%%%%%%%%%%%%%%%%%%%%%%%%%
%%%%%%%%%%%%%%%%%%%%%%%%%%%%%%%%%%%%%%%%%%%%%%%%%%%%%%%%%%%%%%%%%%%%%%%%%%%%%%%


\zzz{Linear Equations}
  \1 system is inconsistent if it has no solution
  \1 a system must have either none, one, or infinitely many solutions
  \1 system is called Homogeneous if all the constant terms (right sides) are 0
  \1 if coefficient matrix is invertible, there is one unique solution for the system

\zzz{Gaussian Elimination}
  \1 first put system into matrix form
  \1 Elementary Operations:
    \2 multiply a row by a nonzero scalar
    \2 add row multiplied by scalar to another row
    \2 swap rows
  \1 Elementary Operations don't change any of the solutions
  \1 try to make it into a diagonal matrix

\zzz{Row-Echelon form}
  \1 all leading entries $\not= 0$, leading entries shift right as you go down
  \1 from RE form you can use back substitution to get solutions
  \1 leading entry of row: first non-zero element of row
  \1 if there is any zero row, then the solution has a free variable

\zzz{Reduced Row-Echelon form}
  \1 same as RE form, but all leading entries $=1$, each column with a leading entry is zeros everywhere else
    \2 this isn't always the identity matrix; some columns could be missing leading entries entirely

\zzz{Gauss-Jordan reduction}
  \1 use Gaussian Elimination to get matrix into Reduced Row-Echelon form
  \1 if there are columns without leading entries, those are free variables
  \1 take each row, transform back to equation, get solution

\zzz{Matrix}
  \1 matrices $A,B \in M_{m,n}(R)$
  \1 addition, scalar multiplication work like vectors
    \2 addition is only defined when matrix dimensions match
  \1 \textbf{Dot Product:} $x*y = x_1 y_1 + x_2 y_2 + \cdots + x_n y_n = \sum_{k=0}^{n} x_k y_k$
  \1 \textbf{diagonal matrix:} only diagonal elements are non-zero
  \1 identity matrix: $I$, diagonal of $1$'s
    \2 $AI=A$ for any A and properly sized $I$
  \1 \textbf{Transpose:} just swap the rows and columns
    \2 represented as $A^T$
    \2 if $A=A^T$ then $A$ is symmetric

\zzz{Multiplication:} $A_{m \times n}, B_{n \times p}, C=AB$
  \1 $c_{i,j} = \sum_{k=1}^{n} a_{i,k}b_{k,j}$
  \1 each element in $C$ is the dot product of that row in $A$ and column in $B$
  \1 result has as many rows as $A$ and columns as $B$
  \1 $AB$ is only the same size as $BA$ if they're both square; since the size depends on the matching dimensions of $A$ and $B$
  \1 if $AB=BA$ then $A$ and $B$ commute
  \1 $AB(C) = A(BC)$
  \\ $(A+B)C = AC+BC$
  \\ $C(A+B) = CA+CB$
  \\ $(rA)B = A(rB) = r(AB)$

\zzz{Inverse Matrix}
  \1 $AA^{-1} = A^{-1}A = I$
  \1 no inverse: singular or non-invertible
  \1 inverse exists: non-singular or invertible
  \1 zero matrix is singular
  \1 inverse of a matrix is unique
  \1 inverse distributes over matrix multiplication
  \1 inverse of diagonal matrix: reciprocal of each element
  \1 inverse of $2\times 2$ matrix $[a,b;c,d]$ is $\frac{1}{ac-bd}[d,-c;-b,a]$
  \1 find inverse:
    \2 convert $A$ to Reduced Row-Echelon form
    \2 apply those same ordered Elementary Operations to $I$ to get $A^{-1}$
    \2 (you can do these two steps at the same time)
  \1 \textbf{Elementary Matrix:} any matrix reachable by applying Elementary Operations to $I$

\zzz{These are Equivalent}
  \1 $A$ is invertible
  \1 $\det(A) \not= 0$
  \1 $x=0$ is the only solution to the equation $Ax=0$
  \1 $Ax=b$ has a unique solution for any column vector $b$
  \1 Row-Echelon form of $A$ has no zero rows
  \1 Reduced Row-Echelon form of $A$ is $I$
  \1 the rows/columns of $A$ are linearly independent
  \1 the columns of $A$ form a Basis of $R^n$


\zzz{Determinants}
  \1 determinant of singleton matrix is single value
  \1 determinant of $2\times 2$ matrix $[a,b;c,d]$ is $ac-bd$
  \1 determinant of diagonal matrix is product of diagonal entries
    \2 same for upper, lower triangular
  \1 determinant of larger matrix can be broken down by a row or column:
    \2 for each element $a_{i,j}$, take $a_{i,j}$ times the determinant of  the (smaller) matrix formed by leaving out row $i$, column $j$
    \2 and use the proper sign by the alternating method
  \1 Elementary Operation Axioms:
    \2 D1: multiply row by $r$ $\rightarrow$ multiply det by $r$
    \2 D2: add scalar multiple of one row to another $\rightarrow$ same det
    \2 D3: swapping rows of matrix $\rightarrow$ det changes sign
    \2 D4: $\det(I) = 1$
    \2 C1: if $A,B$ are square and $A$ is obtained by applying Elementary Operations to $B$, then $\det(A)=0$ iff $\det(B)=0$
    \2 C2: $\det(B)=0$ whenever $B$ has a zero row
    \2 C3: $\det(A)=0$ iff $A$ is not invertible
  \1 Cramer's rule: explicit formula for solution to system of linear equations, using determinants. Not very useful because determinants.

\zzz{Wronskian}
  \1 to show linear independence of functions in $C^{\infty}(R)$ (continuously differentiable functions)
  \1 $W(f_1,f_2,f_3)(x) = $
    \begin{tabular}{|l l l|}
      $f_1(x)$ & $f_2(x)$ & $f_3(x)$ \\
      $f'_1(x)$ & $f'_2(x)$ & $f'_3(x)$ \\
      $f''_1(x)$ & $f''_2(x)$ & $f''_3(x)$ \\
    \end{tabular}
      \2 functions in rows, derivatives down columns
  \1 if $W(x)$ is not identically $0$, then the functions are linearly independent
  \1 alternatively (1): take derivatives of the top row until you get something that you can work with to solve
  \1 alternatively (2): start with $af_1(x)+bf_2(x)+cf_3(x)=0$ and show that the only solution is $a=b=c=0$

%%%%%%%%%%%%%%%%%%%%%%%%%%%%%%%%%%%%%%%%%%%%%%%%%%%%%%%%%%%%%%%%%%%%%%%%%%%%%%%
%%% post exam 1

\zzz{Basis}
  \1 every vector space has a Basis
  \1 it's like a coordinate system
  \1 Basis = minimum spanning set = maximum set of linearly independent vectors
  \1 can get one by adding linearly independent vectors to a too-small set or removing linearly dependent ones from a too-large one
  \1 \textbf{Dimension:} ($\dim(V)$) number of basis vectors for a vector space
    \2 if $\dim(V)<\infty$ then every basis of $V$ is the same size

\zzz{Matrix Spaces}
  \1 matrix $M_{m,n}$:
  \1 \textbf{Row Space:} subspace of $R^n$ spanned by rows of $M$
    \2 \textbf{Rank:} = dimension of row space (number of linearly independent rows)
    \2 in Row-Echelon form, all non-zero rows are linearly independent
  \1 \textbf{Column Space:} subspace of $R^m$ spanned by columns of $M$
  \1 \textbf{Null Space:} $N(A)$: all $x$ such that $Ax=0$
    \2 aka kernel
    \2 solution set of homogeneous equations with coefficients $A$
    \2 $N(A)$ is subspace of $R^n$
    \2 Nullity $=\dim(N(A))=$ number of free variables
  \1 for any matrix, rank + nullity = number of columns

\zzz{Change of Basis (Coordinates):}
  \1 basis $B_1=\{u,v\}$ of $R^2$
  \1 change basis of $(x,y)$: find $r_1,r_2$ such that $(x,y) = r_1v + r_2u$
  \1 \textbf{Transition Matrix} $T=(u^T,v^T)$ has columns of $u$ and $v$, and maps $B_1\rightarrow R^2$
    \2 that is, $T\cdot(^{x'}_{y'}) = (^x_y)$ when $(x,y)=x'u+y'v$
    \2 inverse works: $T^{-1}\cdot(^x_y) = (^{x'}_{y'})$
  \1 transition between general bases:
    \2 basis $B_1,B_2$ with transition matrix $T_1,T_2$
    \2 $f(x): B_1 \rightarrow B_2 = T_2^{-1}\cdot T_1 \cdot x $
    \2 $f(x): B_2 \rightarrow B_1 = T_1^{-1}\cdot T_2 \cdot x $
    \2 more generally, find one basis's coordinates in terms of another's, but then you'll need to solve $n^2$ equations

\zzz{Linear Relations}
  \1 additivity:  $L(ax+by) = aL(x)+bL(y)$
  \1 homogeneity: $L(ax) = aL(x)$
  \1 kernel: all $v$ such that $L(v)=0$
  \1 range: all possible output values

\zzz{Least Squares}
  \1 express as overdetermined relation $Ax = b$ (where there are more rows than variables)
  \1 Then left-multiply both sides by $A^{-1}$, getting $A^{-1}Ax=A^{-1}b$
  \1 the resulting equation will be fully determined, so solve like normal
  \1 at the end, you get values for x, which are the needful least squares coefficients

\zzz{Orthogonality}
  \1 vectors are orthogonal if their dot product is 0
    \2 the zero vector (and only the zero vector) is orthogonal to itself
  \1 \textbf{Orthogonal Compliment} sets (or subspaces) of vectors are orthogonal if every combination from the two is orthogonal
    \2 $R^3$: a line is orthogonal to a plane, and vice versa
  \1 Orthonormal: vectors that are orthogonal and unit length
  \1 any vector $x$ can be broken into $x=p+o$ where $p$ and $o$ are orthogonal, and $p$ is parallel to a known $y$
    \2 $p = \frac{x\cdot y}{y\cdot y}y$
    \2 $o = x - p$

\end{outline}
\end{flushleft}
\end{multicols*}
\end{document}


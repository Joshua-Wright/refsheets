% (c) Copyright 2015 Josh Wright
\documentclass{article}
\usepackage{mathrsfs,amsmath,amsthm,latexsym,paralist}
\usepackage{mathtools} 
\usepackage{amssymb}   % for \varnothing, \therefore
\usepackage{centernot} % for \centernot
\usepackage{geometry}  % for margins
\usepackage{outlines}  % for outline
\usepackage{multicol} % for multicolumn layout
\geometry{letterpaper, landscape, margin=0.5in}
\newcommand{\upspace}{\vspace{0px}}
\newcommand{\bksp}{\hspace{-3px}}
% \newcommand{\zzz}[1]{\upspace \0 \subsection*{#1:} \addcontentsline{toc}   {subsection}{#1} }
\newcommand{\zzz}[1]{\upspace \0\bksp{\textbf{#1:}}\addcontentsline{toc}{subsection}{#1}}
\newcommand{\zza}[1]{\upspace \1\bksp\textbf{#1:}\addcontentsline{toc}{subsubsection}{#1}}
\newcommand{\aaa}{\upspace \1\bksp}
\newcommand{\bbb}{\upspace \2\bksp}
\newcommand{\ccc}{\upspace \3\bksp}
\newcommand{\ddd}{\upspace \4\bksp}
% makes second-level itemize bullets instead of dashes
\renewcommand\labelitemii{\labelitemi}	

\begin{document}
\allowdisplaybreaks
\begin{multicols*}{2}
\begin{outline}[compactitem]
\noindent
\0 notes.txt
\noindent \\ 
\noindent \\ 
\0 Flip-flop:
\1     Setup time: minimum time from when input D is stable to rising clock edge
\2         input must propagate through internal gates
\1     Hold time: minimum time for D to remain stable after rising clock edge
\2         input must propagate through internal feedback gates
\0 D latch:
\1     sets memory by input whenever clock is high
\1     (not on rising edge of clock)
\0 D flop flop
\1     sets value on rising clock edge
\2         and thus, only once per clock cycle
\1     internally it is two d-latches chained together:
\2         the first with an inverter in front of it's clock
\2         the second without
\1     value of D is available at Q immediately after rising clock edge
\0 T flip flop
\1     on rising edge of clock toggles state if T (input) is high
\1     if T is low, keeps current value
\1     IS edge triggered
\noindent \\ 
\0 shift register
\1     called n-bit shift register
\2         n is how many bits it stores
\1     accepts one bit a a time as input
\2         (this is serial)
\1     outputs all bits at the same time, one pin for each bit
\1     also has a out pin for chaining multiple together
\1     internally it is a chain of flip flops
\2         previous output goes to next input, so the data shifts down the line
\0 rotate register
\1     a shift register with it's output connected to it's input
\1     rotates the stuff in it's buffer
\0 universal shift register
\1     shift register with lots of other functions
\1     e.g. clear, clock, shift, right, left, load, read, control
\noindent \\ 
\0 shifter
\1     just a different combination of wires that moves the bits over
\1     can be designed to map one of the edge bits to the filler bits
\0 barrel shifter
\1     shifter that can shift any arbitrary number of bits
\1     barrel shifter with n shifters can shift between 0 and (2\^n-1) bits
\1     e.g. 3 shifters goes 0-7
\2         and uses shifters of 1, 2, and 4 bits
\noindent \\ 
\0 fraction multiplication approximation in binary
\1     compute A/3, error < 5%
\2         each numerator is picked using:
\3             numerator = (2\^x / desired\_devisor)
\2         1/3 =~ 1/4,   error: (0.25      - 1/3) = 0.08 -> 25   %
\2         1/3 =~ 3/8,   error: (0.375     - 1/3) = 0.04 -> 12.5 %
\2         1/3 =~ 5/16,  error: (0.3125    - 1/3) = 0.02 ->  6   %
\2         1/3 =~ 11/32, error: (0.0.34375 - 1/3) = 0.01 ->  3   %
\1     11/32:
\2         (8+2+1)/32
\2         (2\^3 + 2\^1 + 2\^0 )/2\^5
\2         A/3 =~ ( (C<<3) + (C<<1) + (C<<0) )<<5
\noindent \\ 
\0 counter
\1     n-bit counter has n bits
\2         one design uses n-bit increment-er and n-bit register
\1     designed to roll over, so that you can chain multiple together
\2         has an 'overflow' bit for that reason
\2         overflow is bitwise AND of all output bit numbers
\3             (AND is highest representable number)
\2         must have a count pin separate from clock pin
\3             so that they can be on the same clock when chained
\1     down-counter counts down instead of up
\2         it's overflow is bitwise NAND of all register bits
\1     limit a counter to any arbitrary value:
\2         make circuit that matches that binary value (outputs 1 only when it gets that value)
\2         wire the output of that circuit to the reset pin of the counter
\0 bidirectional counter
\1     must have different carry bit
\1     when counting down, you want NOR all inputs as carry (matches 0)
\noindent \\ 
\0 parallel vs serial
\1     parallel
\2         n-bits wide
\2         n bits move every clock cycle
\2         the bus is n wires plus the clock
\2         usually slower clock than serial
\1     serial
\2         one bit wide
\2         one bit moves every clock cycle
\1     you can convert serial data to parallel using a shift register and clock divider
\0 serial addition
\1     add a number bit by bit
\1     you store the sum in a shift register because that can accept serial input
\1     slower than parallel adder, but uses far fewer gates
\noindent \\ 
\0 register file
\1     pins: clock, w\_data, w\_addr, w\_en, r\_data, r\_addr, r\_en
\1     more than one register in one chip
\1     writing
\2         w\_data is wired to all registers
\3             (buffers are used to make sure the signal doesn't get weak from length)
\2         w\_addr and w\_en go to a decoder which enables one of the registers
\1     reading
\2         the output of every register is wired to the data in port of a driver
\2         r\_addr is wired to a decoder
\2         the output of that decoder is wired to the enable port of each driver
\2         the output of every driver is directly connected to r\_data
\2         this is safe because the decoder ensures that only one of the drivers is ever active
\noindent \\ 
\0 clock divider:
\1     used to take a fast clock input and output a slower output
\1     use a count up counter
\1     if it's a power of 2, you can use an n-bit counter and tc (clock overflow) as clock output
\1     if it's not a power of 2:
\2         up counter
\3             add logic to reset the counter when it reaches the desired number of ticks
\3             connect divided clock output to that reset
\2         down counter
\3             set the load input to the desired number of ticks (as constant inputs)
\3             wire reset to load
\3             connect divided clock output to that reset
\1     the pulse width of the clock output may be different than you expect
\2         it will be right for the power of 2 case
\2         but for the other case, it is probably wrong
\1     there will be a phase difference that is very difficult to fix
\2         due to gate delay
\noindent \\ 
\0 clock timing:
\1     to find maximum clock speed, use the formula:
\2         T = (clk->Q) + (max Q->D) + setup time + skew
\3             T  clock period
\4                 1/T = clock frequency
\3             clk->Q: delay of one flip-flop
\4                 interval of time after clock rising edge where Q has correct value
\3             max Q->D: most delayed path from ANY flip-flop output to ANY flip-flop input
\4                 the input/output don't have to be on the same flip-flop
\3             setup time:
\4                 length of time after the clock rising edge that D must remain the same value in order for the flip-flop to function
\3             skew: optional, clock skew of the circuit
\4                 (if not given, it is 0)
\noindent \\ 
\0 metastability
\1     happens when
\2         you have asynchronous input (e.g. human pres-able button)
\2         you break the setup time or hold time
\1     when output is metastable it is somewhere between 0 and 1
\2         you can't predict where it is
\1     fix for button:
\2         wire it into a synchronizer flip-flop and then to the rest of your circuit
\3             (because flip-flop uses the same clock as the rest of your circuit)
\2         maybe more than one synchronizer flip-flop for a very clean signal
\3             (at the expense of one clock cycle delay each)
\0 glitching
\1     outputs that are initially not stable
\2         e.g. signal flip flops a little before settling on the correct value
\1     happens because not all inputs get to the logic at the same time due to gate delay and stuff
\1     fix by putting a d-latch or flip-flop after the output
\2         called registered output
\2         this way the output isn't seen until the next clock cycle
\2         (the output is assumed to be stable by then. If it isn't, you probably need a longer clock cycle)
\2         you can combine that flip flop into the state register
\noindent \\ 
\noindent \\ 
\0 FSM: Finite State Machine
\1     usually store the state in a state register
\2         could also use a collection of flip flops, shift register or whatever
\1     to prove that two edges are disjunct:
\2         AND together the boolean expressions, and simplify
\2         you should end up at 0 because it is never possible to satisfy both transitions
\1     to prove that one condition is always satisfiable
\2         OR together all boolean expressions and simplify
\3             you should get true because one of them is true for every input
\1     normal state encoding
\2         assign a number to each state
\2         use the binary representation of that number
\2         allows for the smallest register possible
\1     one-hot encoding
\2         for n states, use a n-bit register
\2         each state is n binary bits with one 1 and the rest 0
\2         may allow for simpler next-state logic
\0 FSM state reduction
\1     group states of FSM by what they output
\1     then, for each input:
\2         sub-group the groups by next state according to every combination of that input
\1     after all this, if 2 states are always in the same group, they are duplicates
\0 FSM -> circuit: (will be on final)
\1     1. assign each state a value for the state register to hold
\2         don't worry about optimizations yet
\1     2. make truth table
\2         inputs:  current state, inputs to FSM
\2         outputs: next state, output
\1     3. design circuit using truth table
\2         current state + inputs -> next state logic
\1     if you're asked to match a bit pattern, it is easier to use a shift register
\1     (or make one out of d-flip-flops)
\0 FSM reducing states:
\1     group by z-value
\1     sub-group by arrows pointing into them
\1     and then by arrows pointing out
\1     then any states in the same group are duplicates
\0 Mealy vs Moore FSM:
\1     Moore 
\2         output is dependent on current state only
\3             when input changes, output does not change until next clock cycle
\2         output logic can be separated completely from next state logic
\3             (output logic is then not dependent on inputs)
\1     Mealy
\2         outputs on arrows
\2         output is dependent on state and inputs
\3             when input changes, output changes asynchronously
\3             (potentially not to the correct value)
\2         means outputs are not lined up with clock edges
\2         Mealy and Moore FSMs can be combined
\noindent \\ 
\0 HLFSM: High Level Finite State Machine
\1     compared to normal FSM:
\2         multi-bit input/output
\2         local storage
\2         arithmetic operations
\3             (FSM can only do boolean, since everything is 1 bit anyway)
\2         a state transition can be a more complex expression
\3             e.g. (in\_reg > 5)
\1     convention:
\2         := for assignment
\2         == for equality comparison
\2         '0' for single bit value
\2         0 for integer value
\2         "00" for multi-byte value
\3             multi-byte outputs must be local storage
\noindent \\ 
\0 RTL Design: Register Transfer Level Design
\1     high-level description of the logic involved
\1     uses multi-byte I/O and generic blocks
\2         multi-byte I/O is like arrays
\1     like a block diagram
\2         too complex for a normal FSM, so design with a HLFSM
\noindent \\ 
\noindent \\ 
\0 RAM memory types
\1     register file
\2         discussed previously
\1     SRAM: Static RAM: Random Access Memory
\2         uses 2 inverters back-to-back
\1     DRAM: Dynamic RAM
\2         uses special on-chip capacitors to store data
\2         data is encoded in the charge of the capacitor, so it is possible to store multiple bits per cell
\2         requires special chip design, so it's usually a separate chip as everything else
\2         charge value can decay, so it must be refreshed
\3             (usually taken care of by the chip firmware)
\1     size:
\2         register file: lots of transistors/cell (least dense)
\2         SRAM: 6 transistors/cell
\2         DRAM: 1 transistor/cell (most dense)
\1     Speed:
\2         register file: fastest
\2         SRAM
\2         DRAM: slowest
\1     RAM has setup/hold times just like flip-flops
\0 ROM: Read Only Memory
\1     non-volatile
\1     can be very fast
\1     low power
\1     does not need to be refreshed (like DRAM)
\1     PROM: Programmable Read Only Memory
\2         can be written to only once
\2         implemented using fuses or ant-fuses
\1     EPROM: Erasable Programmable Read Only Memory
\2         written to using higher-than-normal voltage
\2         erased by exposure to UV light
\3             meaning chip must have a window
\1     EEPROM: Electrically Erasable Programmable Read Only Memory
\2         EPROM than can be erased electrically
\noindent \\ 
\0 MEMS: Micro Electrical Mechanical System
\1     tiny chip-sized devices that aren't necessarily gates or stuff
\1     e.g. gyroscopes
\noindent \\ 
\0 Queue
\1     FIFO: First in, First Out
\1     visualize as a circle
\1     contains limited max size elements
\1     2 pointers: front and rear
\1     push (write):
\2         write to address rear
\2         increment rear
\1     pop (read):
\2         read from address front
\2         increment front
\2         does not really destroy content, it is still accessible
\1     if front or rear reaches size:
\2         next value should be 0
\1     if rear == front:
\2         could mean full or empty
\2         depends if this happens after read or write
\2         after write: it's full
\2         after read: it's empty
\0 \#\#\# making a FSM for a Queue is a good exercise
\1     should probably do this as studying for the exam
\noindent \\ 
\0 RTL tradeoffs
\1     Latency
\2         time from data input to data output in seconds
\1     Throughput
\2         how many jobs can you input per second
\1     pipelining
\2         do things in stages, like an assembly line
\2         helps with throughput
\2         does not help with latency
\2         put a pipeline register in in-between steps
\2         doesn't make task faster, but you can start a new task before the previous one finishes
\1     concurrency:
\2         do things side by side
\2         helps with throughput but not latency
\1     it is sometimes possible to do both concurrency and pipelining
\1     component allocation:
\2         choice of components used to implement design
\2         example: if you have 2 states that need multiplication:
\3             it might be more compact to have them use the same multiplier and simply MUX the inputs
\3             slightly more delay, many fewer gates
\1     operator binding:
\2         how you configure the MUXes in compact allocation
\2         if you map them efficiently you can reduce delay
\2         e.g. if one input is used in more than one scenario, arrange the inputs such that that one always has the same port
\3             and then, you don't need a MUX there at all
\1     operator scheduling:
\2         introduce new states to preform operations
\3             downside is more states
\3             upside is states do fewer things each
\3             (which means fewer gates)
\2         e.g. to reduce the number of multipliers required
\2         can reduce the longest path through the circuit which allows you to increase the clock speed
\1     other optimizations:
\2         serial vs concurrent computation
\3             datapath: carry-ripple (serial) vs carry-lookahead adder (parallel)
\noindent \\ 
\0 chip design types
\1     full custom
\2         exactly what it sounds like, gate for gate design
\2         very hard to make
\2         takes a long time from design to mass production
\2         cheapest per chip in large quantities
\2         most expensive per chip if you're making only one
\3             (very much in the realm of "Why would you do that?!?")
\1     standard cell ASIC: Application Specific Integrated Circuit
\2         has pre laid out cells
\2         each cell can be made into a gate
\2         you must still make custom mats to fabricate the circuit
\2         vs full custom
\3             easier
\3             bigger and slower
\3             more expensive per chip
\1     gate array
\2         structured ASIC
\2         gates are already on the chip, they just need to be wired together
\2         vs standard cell ASIC
\3             faster and easier to manufacture
\3             bigger and slower
\3             more expensive per chip
\2         some use only NAND or NOR gates
\3             since both of those are Turing complete
\1     FPGA: Field Programmable Gate Array
\2         like a gate array that you can reprogram on the fly
\3             at least the original ones were.
\3             modern ones use lookup tables
\2         vs gate array
\3             easier
\3             slower
\3             more expensive
\3             bigger
\4                 no non-recurring costs
\1     Single IC's
\2         like the stuff we did in the first few labs
\2         does not scale at all
\1     SPLD: Single Programmable Logic Device
\2         predates FPGAs
\2         prefabricated IC with large AND and OR structure
\2         has programmable nodes before each AND gate
\3             could be fuse based (not reusable)
\3             or memory based (reusable)
\2         used to implement simple logic
\noindent \\ 
\0 how a FPGA works
\1     \#\#\# TODO
\1     LUT = Look Up Table
\2         MUX + memory
\2         example: 3 binary inputs = 8 combinations
\2         use 8-1 MUX map 8 pre-computed input values
\2         if a circuit uses more inputs than a LUT can handle, you must cascade to more than one LUT
\3             (just like you do with gates)
\3             this is more efficient than larger memories anyway
\2         it is common for a LUT to have fewer than it's maximum inputs
\3             thus the extra data is wasted, but that isn't a big deal
\2         you can also have LUTs with more than one bit outputs
\1     SM: Switch Matrices
\2         AKA programmable interconnect
\2         the way that LUTs are connected together
\2         also MUX+memory based
\2         each output is MUXed to an input
\1     CLB: Configurable Logic Block
\2         LUT + flip-flop + MUX + memory
\2         memory programs MUX to output stored (flip flop) value or logic value
\1     chip layout
\2         hundreds of  thousands of SMs and CLBs in a grid with wires connecting adjacent blocks
\0 programming a FPGA
\1     all memory of all blocks is connected sequentially as one big shift register
\2         known as a scan chain
\1     you shift in your configuration (code) that does stuff and then the FPGA is programmed
\noindent \\ 
\noindent \\ 
\0 \#\#\# TODO: processor internals
\end{outline}
\end{multicols*}
\end{document}

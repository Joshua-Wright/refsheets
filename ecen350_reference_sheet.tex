% Josh_Wright_Resume.tex
% (c) Copyright 2015 Josh Wright
\documentclass[12pt]{article}
\usepackage{verbatim}
% \usepackage{syntonly}
\usepackage{ragged2e}
\usepackage{geometry}
\usepackage{enumitem} % for longenum
\usepackage{setspace}
\usepackage{hyperref}
\usepackage{tabularx}
\usepackage{graphicx}
\usepackage{outlines} % for outline
\usepackage{paralist} % for compactitem (compact itemize)
\usepackage{multicol} % for multicolumn layout
\geometry{letterpaper, margin=0.5in, top=0.3in}
% \geometry{letterpaper, margin=0.5in, top=0.35in, left=1.5in}

\graphicspath{ {images/} }

\begin{document}
\begin{spacing}{0.8}
% \linespread{0.5}
\begin{center}
ECEN350 Computer Architecture Ref Sheet
\hfill \textcopyright \space Josh Wright 2015 \hfill
Last Updated: \today
\end{center}

%%%%%%%%%%%%%%%%%%
%% main section %%
%%%%%%%%%%%%%%%%%%
\begin{multicols*}{2}
\begin{flushleft}
\newlist{longenum}{itemize}{5}
\setlist[longenum,1]{nosep,leftmargin=0.4cm,labelwidth=0px,align=left,label=$\bullet$}
\setlist[longenum,2]{nosep,leftmargin=0.4cm,labelwidth=0px,align=left,label=$\ast$}
\setlist[longenum,3]{nosep,leftmargin=0.4cm,labelwidth=0px,align=left,label=-}
\setlist[longenum,4]{nosep,leftmargin=0.4cm,labelwidth=0px,align=left,label=>}
\setlist[longenum,5]{nosep,leftmargin=0.4cm,labelwidth=0px,align=left,label=@}
% \begin{outline}[compactitem]
\begin{outline}[longenum]

%%%%%%%%%%%%%%%%%%%%
%% spacing config %%
%%%%%%%%%%%%%%%%%%%%
% just in case I need even more space
\newlength{\upspacelength}
\setlength{\upspacelength}{0px}
\newcommand{\upspace}{\vspace{\upspacelength}}
% section titles
\newcommand{\zzz}[1]{\upspace \0 \textbf{#1} }
% \newcommand{\zzz}[1]{\0 \hspace{-1.25in} \textbf{#1} \vspace{-10px} }
% makes second-level itemize bullets instead of dashes
% \renewcommand\labelitemii{\labelitemi}
% redefine the sub-headings to inject our space-saver
\let\oldOne\1\let\oldTwo\2\let\oldThree\3\let\oldFour\4
\renewcommand{\1}{\upspace \oldOne  }
\renewcommand{\2}{\upspace \oldTwo  }
\renewcommand{\3}{\upspace \oldThree}
\renewcommand{\4}{\upspace \oldFour }
\small

% \zzz{Hex}
%   \\
%   \begin{tabular}{|r r r|}\hline
%    0 & \verb|0x0| & \verb|0000| \\
%    1 & \verb|0x1| & \verb|0001| \\
%    2 & \verb|0x2| & \verb|0010| \\
%    4 & \verb|0x4| & \verb|0100| \\
%    8 & \verb|0x8| & \verb|1000| \\ \hline
%   \end{tabular}

\zzz{Registers}
  \\
  \begin{tabular}{|l r l l|}\hline
  n & $_{10}$ & hex & bin   \\\hline
  \verb|$0 | & \verb| 0| & \verb|0x00| & \verb|00000| \\ \hline
  \verb|$at| & \verb| 1| & \verb|0x01| & \verb|00001| \\ \hline \hline
  \verb|$v0| & \verb| 2| & \verb|0x02| & \verb|00010| \\ \hline
  \verb|$v1| & \verb| 3| & \verb|0x03| & \verb|00011| \\ \hline \hline
  \verb|$a0| & \verb| 4| & \verb|0x04| & \verb|00100| \\ \hline
  \verb|$a1| & \verb| 5| & \verb|0x05| & \verb|00101| \\ \hline
  \verb|$a2| & \verb| 6| & \verb|0x06| & \verb|00110| \\ \hline
  \verb|$a3| & \verb| 7| & \verb|0x07| & \verb|00111| \\ \hline \hline
  \verb|$t0| & \verb| 8| & \verb|0x08| & \verb|01000| \\ \hline
  \verb|$t1| & \verb| 9| & \verb|0x09| & \verb|01001| \\ \hline
  \verb|$t2| & \verb|10| & \verb|0x0a| & \verb|01010| \\ \hline
  \verb|$t3| & \verb|11| & \verb|0x0b| & \verb|01011| \\ \hline
  \verb|$t4| & \verb|12| & \verb|0x0c| & \verb|01100| \\ \hline
  \verb|$t5| & \verb|13| & \verb|0x0d| & \verb|01101| \\ \hline
  \verb|$t6| & \verb|14| & \verb|0x0e| & \verb|01110| \\ \hline
  \verb|$t7| & \verb|15| & \verb|0x0f| & \verb|01111| \\ \hline
  \end{tabular}
  \begin{tabular}{|l r l l|}\hline
  \verb|$s0| & \verb|16| & \verb|0x10| & \verb|10000| \\ \hline
  \verb|$s1| & \verb|17| & \verb|0x11| & \verb|10001| \\ \hline
  \verb|$s2| & \verb|18| & \verb|0x12| & \verb|10010| \\ \hline
  \verb|$s3| & \verb|19| & \verb|0x13| & \verb|10011| \\ \hline
  \verb|$s4| & \verb|20| & \verb|0x14| & \verb|10100| \\ \hline
  \verb|$s5| & \verb|21| & \verb|0x15| & \verb|10101| \\ \hline
  \verb|$s6| & \verb|22| & \verb|0x16| & \verb|10110| \\ \hline
  \verb|$s7| & \verb|23| & \verb|0x17| & \verb|10111| \\ \hline \hline
  \verb|$t8| & \verb|24| & \verb|0x18| & \verb|11000| \\ \hline
  \verb|$t9| & \verb|25| & \verb|0x19| & \verb|11001| \\ \hline \hline
  \verb|$k0| & \verb|26| & \verb|0x1a| & \verb|11010| \\ \hline
  \verb|$k1| & \verb|27| & \verb|0x1b| & \verb|11011| \\ \hline
  \verb|$gp| & \verb|28| & \verb|0x1c| & \verb|11100| \\ \hline
  \verb|$sp| & \verb|29| & \verb|0x1d| & \verb|11101| \\ \hline
  \verb|$fp| & \verb|30| & \verb|0x1e| & \verb|11110| \\ \hline
  \verb|$ra| & \verb|31| & \verb|0x1f| & \verb|11111| \\ \hline
  \end{tabular}
  \1 callee saved registers: \verb|$s0-$s7, $sp, $gp, $fp|
    \2 save parent's value at beginning of function
  \1 caller saved registers: basically all the others
    \2 save your value before calling subroutine
  \1 general format is to list destination first, then operands

\zzz{Clock Rate}
\\
\begin{tabular}{|r r|}    \hline
  period   & rate    \\ \hline
    1 msec &   1 MHz \\ \hline
  100 nsec &  10 MHz \\ \hline
   10 nsec & 100 MHz \\ \hline
    5 nsec & 200 MHz \\ \hline
\end{tabular}
\begin{tabular}{|r r|}    \hline
    2 nsec & 500 MHz \\ \hline
    1 nsec & 1 GHz   \\ \hline
  500 psec & 2 GHz   \\ \hline
  250 psec & 4 GHz   \\ \hline
  200 psec & 5 GHz   \\ \hline
\end{tabular}

\zzz{Metric Prefixes} \\
\begin{tabular}{|c c l l|}                                   \hline
peta  & P     & $10^{ 15}$ & \hfill 1 000 000 000 000 000 \\ \hline
tera  & T     & $10^{ 12}$ & \hfill     1 000 000 000 000 \\ \hline
giga  & G     & $10^{  9}$ & \hfill         1 000 000 000 \\ \hline
mega  & M     & $10^{  6}$ & \hfill             1 000 000 \\ \hline
kilo  & k     & $10^{  3}$ & \hfill                 1 000 \\ \hline
hecto & h     & $10^{  2}$ & \hfill                   100 \\ \hline
deca  & da    & $10^{  1}$ & \hfill                    10 \\ \hline
one   &       & $10^{ 0 }$ & \hfill       1 \hfill \hfill \\ \hline
deci  & d     & $10^{- 1}$ & 0.1                          \\ \hline
centi & c     & $10^{- 2}$ & 0.01                         \\ \hline
milli & m     & $10^{- 3}$ & 0.001                        \\ \hline
micro & $\mu$ & $10^{- 6}$ & 0.000 001                    \\ \hline
nano  & n     & $10^{- 9}$ & 0.000 000 001                \\ \hline
pico  & p     & $10^{-12}$ & 0.000 000 000 001            \\ \hline
femto & f     & $10^{-15}$ & 0.000 000 000 000 001        \\ \hline
\end{tabular}

\zzz{J format (absolute branching)}
  \1 cannot change the top 4 bits of PC. (\verb|PC[31:28]|)
  \1 range:
    \2 total of $2^{26}$ instructions or $2^{28}$ bytes
      \3 because range is $[0,2^{26}-1]$
    \2 farthest possible next instruction is $2^{26}$ away (if \verb|PC+4| lies at the beginning of a $2^{28}$ byte boundary)
    \2 worst case is you can only jump 1 instruction ahead (if \verb|PC+4| lies at the end of a $2^{28}$ byte boundary)
  \1 conversion:
    \2 instruction stores 26 bits
    \2 right pad with two 0s to get 28
    \2 take the top four bits from current PC to get 32
  \1 mask of top 4 bits: \verb|0xF0000000|
  \1 \verb|target = (PC AND 0xF0000000) OR (addr << 2)|
  % python: j = lambda PC,addr: hex((PC & 0xF0000000) | (addr << 2))

\zzz{Relative Branching}
  \1 range: $[ PC - 2^{17}, PC + 2^{17} - 4 ]$
    \2 that's in bytes. It's a range of $2^{15}-1$ words
    \2 you lose one from the exponent because it's 2's complement
  \1 conversion
    \2 take 16 bit offset, zero pad by 2 (multiply by 4)
    \2 add to PC+4 (next PC)
  \1 \verb|target = (PC + 4) + (addr << 2)|
  \1 due to the \verb|PC+4| thing, if you want to jump back to the same instruction, the immediate value will be -1

\zzz{Endianness}
\\
Value: \verb|0xA0B0C0D0|\\
  \1 
  \begin{tabular}{l r r r r}
  index  & 0 & 1 & 2 & 3 \\
  little & \verb|0xD0| & \verb|0xC0| & \verb|0xB0| & \verb|0xA0| \\
  big    & \verb|0xA0| & \verb|0xB0| & \verb|0xC0| & \verb|0xD0| \\
  \end{tabular}
    \2 Little Endian puts the least significant (littlest) stuff first
  \1 x86 is little endian, MIPS is big endian
  \1 networking is done in big endian

\zzz{Two's Complement}
  \1 $N$ bits can represent a range $[ -2^N, +2^N - 1 ]$
  \1 methods for converting negative values
  \1 method 1:
    \2 start with absolute value
    \2 flip all bits (bitwise not)
    \2 add 1
  \1 method 2:
    \2 use $N+1$ bits ($2^N$ is $N+1$ bits)
    \2 start with absolute value $x$
    \2 find $2^N - x$
    \2 truncate

\zzz{Shifts}
  \1 shift left always fills with 0s
  \1 \textbf{Logical} left shift fills with 0s
  \1 \textbf{Arithmetic} left shift sign-extends
    \2 extends based on far left bit (most significant)
 
\zzz{Assembler}
  \1 Spilling: when a compiler puts a variable in main memory because it's run out of registers
    \2 the variable has spilled to RAM
    \2 inverse is filling
  \1 Object file sections: header; text; data; relocation information; symbol table; debugging information
    \2 Object file is assembled assuming that instructions start at \verb|0x00|. (this is corrected later by the linker)
  \1 Global label can be referenced in any file
    \2 you must declare it global in the file where it is defined, and declare it global again where it's used
    \2 \verb|main| must be global so the linker can find it
    \2 \verb|printf| is global so you can use it (but you must still declare it as global in that file where you use it)
  \1 local label can be referenced in only the current file
    \2 labels are local by default
  \1 \textbf{Symbol Table:} contains all external references
    \2 also lists unresolved references (e.g. printf)
    \2 as far as assembler is concerned, symbol table contains both local and global labels, resolved and unresolved.
    \2 The final assembled object file only contains global labels
  \1 \textbf{Relocation Table:} contains references to all things that depend on absolute addresses
    \2 e.g. all absolute jumps, load address
    \2 these must be changed after loading into memory
    \2 does not contain addresses of labels


%%%%%%%%%%%%%%%%%%%%%%%%%%%%%%%%%%%%%%%%%%%%%%%%%%%%%%%%%%%%%%%%%%%%%%%%%%%%%%%%
%%% exam 2

\zzz{Verilog}
  \1 always block: synthesize to combinational logic iff:
    \2 everything written to is always written exactly once for every case of inputs
    \2 the outputs of the always block depend only on inputs that are in the sensitivity list
    \2 stuff assigned to inside an always block must be declard \verb|reg|
      \3 will be optimized out if it's combinational
  \1 bitwise not is $\sim$
  \1 ternary operator: \verb|cond ? if_true : if_false|
  \1 assignments: \verb|=| is blocking, \verb|<=| is non-blocking
    \2 \verb|=|: happens in order
    \2 \verb|<=|: happens all at once
  \1 case statement: can use \verb|?| to specify `don't care' for some bits
  \1 \verb|`timescale unit/precision|:
    \2 \verb|unit|: 1, 10, or 100, unit either s, ms, us, ps, fs
    \2 \verb|precision|: must be shorter than \verb|unit|

\zzz{Performance}
  \1 execution time = (\# of clock cycles) $\times$ (clock cycle time)
    = (\# of clock cycles)/(clock rate) 
  \1 CPI: Cycles Per Instruction
    \2 effective CPI is just a weighted average (varies by instruction mix)
  \1 instructions per time = CPI / clock rate = CPI * clock period
  \1 compare two systems:
    \2 use instruction latencies and instruction mix to calculate CPI for each setup
    \2 then calculate instructions per time, and do comparison there 

\zzz{IEEE Floating-Point}
  \1 1 bit sign; 8 bit exponent; 23 bit mantissa
    \2 $x = (-1)^s \cdot (1::m) \cdot 2^{e-127}$
  \1 sign: 0 for positive, 1 for negative
  \1 exponent: bias is $-127$
  \1 mantissa: the fractional part; denominator $2^{23}$
    \2 implicit leftmost bit is not stored, only fractional
  \1 conversion: decimal to float:
    \2 start with $x$
    \2 use $\lfloor \log_2 \rfloor$ to express $x$ as $a \cdot 2^b$ where $1\leq a < 2$
    \2 exponent $=127+b$
    \2 mantissa $=(a-1) \cdot 2^{23}$
      \3 round to nearest integer
  \1 conversion: float to decimal:
    \2 real exponent $a = exp-127$
    \2 take exponent as integer $\rightarrow a$
    \2 decimal = $(1 + \frac{a}{2^{23}}) \cdot 2^a$
  \1 calculate mantissa directly: $\frac{x}{2^{\lfloor \log_2(x) \rfloor}} \cdot 2^{23}$
  \1 mantissa the long way:
    \\ take right-of-decimal part and repeatedly multiply by 2. On each iteration, the 1's place is that bit in the mantissa. (starting from leftmost bit)
  \1 quantity of numbers on range $[2^n, 2^{n+1}] = 2^{23} + 1$
    \\ quantity of numbers on range $[2^n, 2^{n+1}) = 2^{23}$
    \2 the $2^{n+1}$ bumps it up because the exponent changes
  \1 next largest float: add $2^{-23}$ to mantissa (assuming exponent doesn't change, i.e. number isn't evenly $2^n$)


\end{outline}
\end{flushleft}
\end{multicols*}
\end{spacing}
\end{document}


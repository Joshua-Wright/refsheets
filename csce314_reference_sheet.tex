% # (c) Copyright 2017 Josh Wright
\documentclass{article}
\usepackage{mathrsfs,amsmath,amsthm,latexsym,paralist}
\usepackage{mathtools} 
\usepackage{bm} 
\usepackage{listings}
\usepackage{amssymb}   % for \varnothing, \therefore
\usepackage{centernot} % for \centernot
\usepackage{geometry}  % for margins
\usepackage{outlines}  % for outline
\usepackage{paralist}  % for compactitem, compactenum
\usepackage{graphicx}
\usepackage{enumitem}
\usepackage{multicol}
\usepackage{minted}
\usepackage{hyperref}
\usepackage{fontspec}
\hypersetup{
  pdftitle={CSCE 314 Reference Sheet},
  pdfauthor={Josh Wright},
  pdfsubject={CSCE 314},
  bookmarksnumbered=true,
  bookmarksopen=true,
  bookmarksopenlevel=1,
  colorlinks=true,
  pdfstartview=Fit,
  pdfpagemode=UseOutlines,
  colorlinks=true,
  linkcolor=blue,
  filecolor=magenta,      
  urlcolor=cyan,
}
%%%%%%%%%%%%%%%%%%%%%%%%%%%%%%%%%%%%%%
%% paper size, orientation, margins %%
%%%%%%%%%%%%%%%%%%%%%%%%%%%%%%%%%%%%%%
% % good for on screen-only viewing
% \def \columncount {2}
% \geometry{letterpaper, landscape, margin=0.25in}
% good for printing
\def \columncount {2}
\geometry{letterpaper, portrait, margin=0.25in, bottom=0.5in}
\setlength{\footskip}{10pt}
\setmonofont{Source Code Pro}


% makes second-level itemize bullets instead of dashes
% \renewcommand\labelitemi{\cdot}
\renewcommand\labelitemi{\hspace{-1in}\tiny$\bullet$}
\renewcommand\labelitemii{\labelitemi}

\newcommand{\codesize}{8.5}
\AtBeginEnvironment{minted}{\fontsize{\codesize}{\codesize}\selectfont}
\newcommand{\java}[1]{{\fontsize{\codesize}{\codesize}\selectfont\mintinline{java}{#1}}}
\newcommand{\haskell}[1]{{\fontsize{\codesize}{\codesize}\selectfont\mintinline{haskell}{#1}}}
% \newenvironment{java}{\ begin{minted}{java}}{\end{minted}{java}}
% \newenvironment{haskell}{\ begin{minted}{haskell}}{\end{minted}{haskell}}

% \newminted{java}{breaklines}
\setminted{breaklines}

\newcommand{\p}{\partial}
\newcommand{\ang}[1]{\left\langle #1 \right\rangle}
\newcommand{\ceil}[1]{\left\lceil #1 \right\rceil}
\newcommand{\floor}[1]{\left\lfloor #1 \right\rfloor}
\newcommand{\prt}[2]{\frac{\partial#1}{\partial#2}}
\newcommand{\grad}{\nabla}
\newcommand{\vrt}[1]{\rotatebox{90}{#1}}
\newcommand{\vrta}[2]{\rotatebox{90}{#1} \rotatebox{90}{#2}}
\newcommand{\vrtb}[3]{\rotatebox{90}{#1} \rotatebox{90}{#2} \rotatebox{90}{#3}}
\newcommand{\vrtc}[4]{\rotatebox{90}{#1} \rotatebox{90}{#2} \rotatebox{90}{#3} \rotatebox{90}{#4}}

\newcommand{\remove}[1]{}
% \newcommand{\remove}[1]{#1}

\newcommand{\tabletextsize}{\large}


% \setlength{\parindent}{0em}
% \setdefaultleftmargin{0em}{0em}{}{}


% super-dense mode:
% \renewcommand{\1}{\upspace $\bullet$\hspace{-0.5em} }
% \renewcommand{\2}{\upspace $\bullet\bullet$\hspace{-0.5em} }
% \renewcommand{\3}{\upspace $\bullet\bullet\bullet$\hspace{-0.5em} }
% \renewcommand{\4}{\upspace $\bullet\bullet\bullet\bullet$\hspace{-0.5em} }

% \newcommand{\zzz}[1]{{\noindent\textbf{#1:}}}
% \renewcommand{\1}{$\bullet$}
% \renewcommand{\2}{$\bullet$}
% \renewcommand{\3}{$\bullet$}
% \renewcommand{\4}{$\bullet$}


\begin{document}
\allowdisplaybreaks
% asterisk makes multicols finish one column before going onto the next
\begin{multicols*}{\columncount}
\setlength{\columnseprule}{0.2pt}
\newlist{longenum}{itemize}{5}
\setlist[longenum,1]{nosep,leftmargin=0.2cm,labelwidth=0pt,align=left,label=$\bullet$}
\setlist[longenum,2]{nosep,leftmargin=0.2cm,labelwidth=0pt,align=left,label=$\ast$}
\setlist[longenum,3]{nosep,leftmargin=0.2cm,labelwidth=0pt,align=left,label=-}
\setlist[longenum,4]{nosep,leftmargin=0.2cm,labelwidth=0pt,align=left,label=>}
\setlist[longenum,5]{nosep,leftmargin=0.2cm,labelwidth=0pt,align=left,label=@}
\begin{outline}[longenum]

\newcommand{\upspace}{\vspace{-0.6pt}\linespread{1}}
% section titles
% makes second-level itemize bullets instead of dashes
\renewcommand\labelitemii{\labelitemi}
% redefine the sub-headings to inject our space-saver
\let\oldOne\1\let\oldTwo\2\let\oldThree\3\let\oldFour\4
\newcommand{\zzz}[1]{\noindent\0\noindent {\textbf{#1:}} \upspace}
\renewcommand{\1}{\upspace \oldOne   \hspace{-6pt}}
\renewcommand{\2}{\upspace \oldTwo   \hspace{-6pt}}
\renewcommand{\3}{\upspace \oldThree \hspace{-6pt}}
\renewcommand{\4}{\upspace \oldFour  \hspace{-6pt}}

\noindent
\textbf{CSCE 314 Reference Sheet} \hfill \textcopyright\space\today \hfill Josh Wright
% \vspace{-0.4cm}


%%%%%%%%%%%%%%%%%%%%%%%%%%%%%%
%% real content starts here %%
%%%%%%%%%%%%%%%%%%%%%%%%%%%%%%

\zzz{General}
  \1 \haskell{newtype Parser a = P (String -> [(a,String)])}
  \1 Predicate: a function that takes one argument and returns a boolean
    \2 if \haskell{pred x == True} then \haskell{x} satisfies predicate \haskell{pred}
  \1 function composition:
\begin{minted}{haskell}
-- the . operator composes functions:
(f . g) x == f (g x)
\end{minted}

\0 useful library functions:
\begin{minted}{haskell}
-- Data.List
nubBy :: (a -> a -> Bool) -> [a] -> [a]
nubBy pred xs = -- unique elements only from xs as determined by pred
nub :: Eq a => [a] -> [a]
nub xs = nubBy (==) a -- unique elements from xs
-- 
words :: String -> [String]
words xs = -- list of whitespace-separated words from xs
--
-- concatenate container of lists
concat :: Foldable t => t [a] -> [a]
-- or for list-of-lists specifically:
concat :: [[a]] -> [a]
concat xs = foldl (++) [] xs
--
-- like concat, but use a function to get the inner lists
concatMap :: (a -> [b]) -> [a] -> [b]
concatMap f xs =  foldr ((++) . f) [] xs
-- 
-- get the longest prefix of xs for which pred is true and also return the rest of the list
span :: (a -> Bool) -> [a] -> ([a], [a])
span pred xs = (takeWhile pred xs, dropWhile pred xs)
-- 
-- repeat a = infinite list of a
repeat :: a -> [a]
repeat x = map (\_ -> x) [1..]
repeat x = [ x | _ <- [1..] ]
-- replicate n a = list of length n repeating a
replicate :: Int -> a -> [a]
replicate n x = map (\_ -> x) [1..n]
replicate n x = [ x | _ <- [1..n] ]
--
-- folds (works on any foldable, not just lists)
foldr :: (a -> b -> b) -> b -> [a] -> b
foldr f z [a,b,c] = a `f` (b `f`  (c `f` z))
foldr f z [a,b,c] = f a $ f b $ f c z
-- combines into z from right to left
-- can potentially work on an empty list if one of the folds does not evaluate it's second argument
foldl :: (b -> a -> b) -> b -> [a] -> b
foldl f z [a,b,c] = ((z `f` a) `f` b) `f` c
foldl f z [a,b,c] = f (f (f z a) b) c
-- evaluates from right to left
-- will not work on infinite list because it must start at the end of the list
-- 
-- these are the same as above, except they take the first two elements for the first application of f
foldr1 :: (a -> a -> a) -> [a] -> a
foldl1 :: (a -> a -> a) -> [a] -> a
= \end{minted}


\zzz{Parsing.hs}
\begin{minted}{haskell}
module Parsing where
import Data.Char
import Control.Applicative (Applicative(..))
import Control.Monad       (liftM, ap)

infixr 5 +++

-- Parser is the name of the type, P is the constructor.
newtype Parser a = P (String -> [(a,String)])

instance Monad Parser where
   return v = P (\inp -> [(v,inp)])
   p >>= f  = P (\inp -> case parse p inp of
                            []        -> []
                            [(v,out)] -> parse (f v) out)

instance Functor Parser where
   fmap = liftM

instance Applicative Parser where
   pure = return 
   (<*>) = ap

failure :: Parser a
failure =  P (\inp -> [])

-- parse any single character
item    :: Parser Char
item    =  P (\inp -> case inp of
                        []     -> []
                        (x:xs) -> [(x,xs)])

parse                         :: Parser a -> String -> [(a,String)]
parse (P p) inp               =  p inp

--Choice
(+++)   :: Parser a -> Parser a -> Parser a
p +++ q =  P (\inp -> case parse p inp of
                         []        -> parse q inp
                         [(v,out)] -> [(v,out)])

--Derived primitives
--------------------
sat      :: (Char -> Bool) -> Parser Char
sat p         =  do x <- item
                    if p x then return x else failure

digit    :: Parser Char
digit         =  sat isDigit

letter   :: Parser Char
letter        =  sat isAlpha

alphanum :: Parser Char
alphanum      =  sat isAlphaNum

lower    :: Parser Char
lower         =  sat isLower

upper    :: Parser Char
upper         =  sat isUpper

char     :: Char -> Parser Char
char x        =  sat (== x)

string   :: String -> Parser String
string []     =  return []
string (x:xs) =  do char x
                    string xs
                    return (x:xs)
many     :: Parser a -> Parser [a]
many p        =  many1 p +++ return []
many1    :: Parser a -> Parser [a]
many1 p       =  do v  <- p
                    vs <- many p
                    return (v:vs)
ident    :: Parser String
ident         =  do x  <- lower
                    xs <- many alphanum
                    return (x:xs)
nat      :: Parser Int
nat           =  do xs <- many1 digit
                    return (read xs)
int      :: Parser Int
int           =  do char '-'
                    n <- nat
                    return (-n)
                  +++ nat
space    :: Parser ()
space         =  do many (sat isSpace)
                    return ()

--Ignoring spacing
------------------
token       :: Parser a -> Parser a
token p     =  do space
            v <- p
            space
            return v

identifier  :: Parser String
identifier  =  token ident

natural     :: Parser Int
natural     =  token nat

integer     :: Parser Int
integer     =  token int

symbol      :: String -> Parser String
symbol xs   =  token (string xs)

--Example: Arithmetic Expressions
expr :: Parser Int
expr  = do t <- term
           do {char '+'
              ;e <- expr
              ;return (t + e)
              }
            +++ return t
term :: Parser Int
term  = do f <- factor
           do char '*'
              t <- term
              return (f * t)
            +++ return f
factor :: Parser Int
factor  = do d <- digit
             return (digitToInt d)
           +++ do char '('
                  e <- expr
                  char ')'
                  return e
eval   :: String -> Int
eval xs = fst (head (parse expr xs))
\end{minted}
  \1 \haskell{sat :: (Char -> Bool) -> Parser Char}
    \2 returns a character if that character satisfies the predicate
  \1 \haskell{digit, letter, alphanum :: Parser Char}
    \2 parses a digit, letter, or alpha-numeric letter respectively
  \1 \haskell{char :: Char -> Parser Char}
    \2 \haskell{char `a'} parses exactly the character \haskell{`a'}
  \1 \haskell{item :: Parser Char}
    \2 parses any character
  \1 similar to above:
    \texttt{
      digit
      letter
      alphanum
      lower
      upper
      string
    }
  \1 \haskell{many :: Parser a -> Parser [a]}
    \2 parses 0 or more instances of \haskell{a} and collects them into a list
  \1 \haskell{many1 :: Parser a -> Parser [a]}
    \2 same as \haskell{many}, but 
  \1 \haskell{(+++)} choice: 
    \2 parse first argument if possible, else parse second argument
    \2 first successfully parsed argument is returned
  \1 \haskell{nat :: Parser Int}
    \2 parse natural number (positive integer)
  \1 \haskell{int :: Parser Int}
    \2 
\begin{minted}{haskell}
(+++) :: Parser a -> Parser a -> Parser a
p +++ q =  P (\inp -> case parse p inp of
                  []        -> parse q inp
                  [(v,out)] -> [(v,out)])
\end{minted}
  \1 \haskell{((>>=))} sequential composition
    \2 \haskell{a >>= b} unboxes monad \haskell{a} into an output \haskell{a0} and then unboxes monad \haskell{b} with input \haskell{a0}
\begin{minted}{haskell}
type Parser a = String -> [(a, String)]
-- implementation for in-class mostly-complete parser `monads'
(>>=) :: Parser a -> (a -> Parser b) -> Parser b
(>>=) p1 p2 = \inp -> case parse p1 inp of
      []  ->  []
      [(v, out)] -> parse (p2 v) out
= \end{minted} 
% for some reason, that equals sign needs to be there for my editor's syntax 
% hilightint to work

  \2 usage:
\begin{minted}{haskell}
doubleDigit :: Parser [Char]
doubleDigit =
  digit >>= \a ->
  digit >>= \b ->
  return [a,b]
-- is equivalent to
doubleDigit' :: Parser [Char]
doubleDigit' = do
  a <- digit
  b <- digit
  return [a,b]
\end{minted}
  \2 \haskell{(>>)} is the same except that it discards the result of the first monad (thus it has signature \haskell{(>>) :: Parser a -> Parser b -> Parser b})

\zzz{Parsing Examples}
  \1 bind and lambda method of parsing:
    \2 parse a number:
  \1 parse arithmetic expressions using do syntax:
\begin{minted}{haskell}
expr :: Parser Int
expr  = do t <- term
           do {char '+'
              ;e <- expr
              ;return (t + e)
              }
            +++ return t

term :: Parser Int
term  = do f <- factor
           do char '*'
              t <- term
              return (f * t)
            +++ return f

factor :: Parser Int
factor  = do d <- digit
             return (digitToInt d)
           +++ do char '('
                  e <- expr
                  char ')'
                  return e

eval   :: String -> Int
eval xs = fst (head (parse expr xs))
\end{minted}

\zzz{Trees}
  \1 represent either a leaf node or some kind of internal node
  \1 arithmetic tree declaration:
\begin{minted}{haskell}
data Expr = Val Int
          | Neg Expr 
          | Add Expr Expr 
          | Mul Expr Expr
\end{minted}
  \1 how to fold over a tree:
\begin{minted}{haskell}
-- exprFold  valF        negF      addF     
exprFold :: (Int->b) -> (b->b) -> (b->b->b) ->
-- mulF        input   output
  (b->b->b) -> Expr -> b
exprFold valF _ _ _ (Val i) = valF i
exprFold valF negF addF mulF (Neg e) 
  = negF  (exprFold valF negF addF mulF e)
exprFold valF negF addF mulF (Add s1 s2) 
  = addF  (exprFold valF negF addF mulF s1)
          (exprFold valF negF addF mulF s2)
exprFold valF negF addF mulF (Mul s1 s2) 
  = mulF  (exprFold valF negF addF mulF s1) 
          (exprFold valF negF addF mulF s2)
\end{minted}
    \2 basically, just collect values into some type \haskell{b} and use supplied functions at each node to fold into single value
    \2 useful for evaluating simple things like:
\begin{minted}{haskell}
-- evaluate an expression
evalExpr'    = exprFold id (\x -> 0 - x) (+) (*)
id --  integers map to integers
(\x -> 0 - x) -- negation
-- everything else is just simple numeric operators
-- 
-- count leaves in a tree
countLeaves' = exprFold (\_ -> 1) id (+) (+)
(\_ -> 1) -- leaf integer node is one node
id -- negation node has only one child, pass on count
(+) (+) -- nodes with two children: add number of leaf grandchildren
\end{minted}


\zzz{HW2: Water Gates}
\begin{minted}{haskell}


waterGate :: Int -> Int
waterGate n =
 length -- number of True's
 $ filter id -- filter just True's
 $ waterGate' n initial -- initial call to helper
 where
  -- start with all gates closed
  initial = replicate n False
  --
  -- flip states
  waterGate' 1 state = map not state 
    -- base case: flip every state
  waterGate' n state = flip n $ waterGate' (n-1) state
  -- otherwise, first get the state for (n-1) and then flip every nth state
  --
  -- flip every nth gate
  flip :: Int -> [Bool] -> [Bool]
  flip 1 xs = map not xs -- flip every gate
  -- flip only gates which index are multiples of n
  flip nth xs = [ if (i `mod` nth == 0) then not x else x
                -- zip each state with it's index
                | (x,i) <- (zip xs [1..])  ]

\end{minted}

\zzz{HW2: Goldbach's Other Conjecture}
\begin{minted}{haskell}
-- check if a number is prime
primeTest :: Integer -> Bool
primeTest 1 = False
primeTest t =  and [ (gcd t i) == 1 | i <- [2..t-1]]

-- all numbers less than n that are double a square
twiceSquares :: Integer -> [Integer]
twiceSquares n = takeWhile (<n) [ 2 *x^2 | x <- [1..]]

-- list of odd numbers
oddList = map (\x -> 2*x + 1) [0..]
-- all odd numbers that are composite (not prime)
allOddComp = [ o | o <- (drop 1 oddList)
                 , not (primeTest o)     ]

-- if a number satisfies conditions for conjecture
-- method: for enough square nubmers, check if n-(that number) is prime
satsConds n = or [ primeTest k | 
                  k <- map (\x->(n-x)) (twiceSquares n) ]

-- find the first number 
goldbachNum = head [ x | x <- allOddComp
                       , not (satsConds x) ]
\end{minted}


\zzz{HW4: Sets}
\begin{minted}{haskell}
type Set a = [a]

a = mkSet [1,2,3,4,5]
b = mkSet [1,2,3]


addToSet :: Eq a => Set a -> a -> Set a
addToSet s a | a `elem` s = s
             | otherwise = a : s

mkSet :: Eq a => [a] -> Set a
mkSet lst = foldl addToSet [] lst

isInSet :: Eq a => Set a -> a -> Bool
isInSet [] _ = False
isInSet [a] b = a == b
isInSet (x:xs) b | x == b = True
                 | otherwise = isInSet xs b

subset :: Eq a => Set a -> Set a -> Bool
subset sub super = and [ isInSet super x | x <- sub  ]

setEqual :: Eq a => Set a -> Set a -> Bool
setEqual a b = subset a b && subset b a

-- instance (Eq a) => Eq (Set a) where
--   a == b = subset a b && subset b a

setProd :: Set a -> Set a -> [(a,a)]
setProd a b = [ (ai,bj) | ai <- a
                        , bj <- b
                        ]

\end{minted}


\zzz{Prev Exam: Run Length Encoding}
\begin{minted}{haskell}
import Parsing
import Data.Char

q4 = do 
  d <- sat isUpper
  e <- char (toLower d)
  f <- many item
  return [d,e]

ones = (map (\_ -> 1) [1..])

myRLE [] = []
myRLE ls = myhelper (zip ones ls)

myhelper [(n,c)] =  [(n,c)] 
myhelper ((n,c):(m,d):rest)
  | (d == c)  = myhelper (((n+m),c):rest)
  | otherwise = (n,c):myhelper ((m,d):rest)
\end{minted}

\zzz{Rock Paper Scissors}
\begin{minted}{haskell}
data RPS = Rock | Paper | Scissors
  deriving (Eq, Show)

rps :: RPS -> RPS -> Int
rps a b | a == b = 0
rps Rock     Scissors = 1
rps Paper    Rock     = 1
rps Scissors Paper    = 1
rps _        _        = 2

rps2 :: RPS -> RPS -> Int
rps2 a b =
  if a == b then 0 else case (a,b) of
    (Rock,     Scissors) -> 1
    (Paper,    Rock)     -> 1
    (Scissors, Paper)    -> 1
    _ -> 2

\end{minted}

\zzz{99 problems}
\begin{minted}{haskell}
-- 9. pack consecutive duplicates into sublists
pack (x:xs) = let (first,rest) = span (==x) xs
               in (x:first) : pack rest
pack [] = []
-- example:
pack [1,2,3,2,2,3] == [[1,1],[2],[3],[2,2],[3]]
\end{minted}



%%% This is here to allow switching the java section to single column, if desired

% \end{outline}
% \end{multicols*}

% \setlength{\columnseprule}{0.2pt}
% \newlist{longenum}{itemize}{5}
% \setlist[longenum,1]{nosep,leftmargin=0.2cm,labelwidth=0pt,align=left,label=$\bullet$}
% \setlist[longenum,2]{nosep,leftmargin=0.2cm,labelwidth=0pt,align=left,label=$\ast$}
% \setlist[longenum,3]{nosep,leftmargin=0.2cm,labelwidth=0pt,align=left,label=-}
% \setlist[longenum,4]{nosep,leftmargin=0.2cm,labelwidth=0pt,align=left,label=>}
% \setlist[longenum,5]{nosep,leftmargin=0.2cm,labelwidth=0pt,align=left,label=@}
% \begin{outline}[longenum]

% \AtBeginEnvironment{minted}{ \fontsize{9}{9}\selectfont }
% \newcommand{\upspace}{\vspace{-0.8pt}\linespread{0}}
% % section titles
% % makes second-level itemize bullets instead of dashes
% \renewcommand\labelitemii{\labelitemi}
% % redefine the sub-headings to inject our space-saver
% \let\oldOne\1\let\oldTwo\2\let\oldThree\3\let\oldFour\4
% \newcommand{\zzz}[1]{\noindent\0\noindent {\textbf{#1:}} \upspace}
% \renewcommand{\1}{\upspace \oldOne   \hspace{-6pt}}
% \renewcommand{\2}{\upspace \oldTwo   \hspace{-6pt}}
% \renewcommand{\3}{\upspace \oldThree \hspace{-6pt}}
% \renewcommand{\4}{\upspace \oldFour  \hspace{-6pt}}


\zzz{{\huge Java}}
  \1 \textbf{Class Invariant}: A logical condition that ensures that an object of a class is in a well-defined state.
    \2 public methods assume that invariant holds before it's called, and makes sure to preserve the invariant property
  \1 garbage collection deals only with memory. You must manage other resources manually, such as concurrent locks, OS file handles, etc...
  \1 anonymous classes are a thing
  \1 inner class:
    class declared inside another class implicitly holds a reference to it's outer class. 
    \\ This means instances of the inner class can use non-static fields/methods of the outer class.
    \\ This is especially handy for callbacks e.g. on android

\zzz{\java{abstract class} vs \java{interface}}
  \1 Interface: all fields are \java{public static final}, all methods are \java{public}
  \1 \java{abstract class} can extend exactly one parent class and implement any number of interfaces
  \1 interface can extend (not implement) any number of interfaces
  \1 abstract class can have constructor that initializes values and whatnot, but interface cannot
    \2 you can't instantiate an abstract class directly, you can only call it's constructor from inside the constructor of a child class. Could still be useful though.
  \1 interfaces and abstract classes can never be instantiated
    \2 references of interface type refer to an instance of a class that implements that interface
    \2 references of abstract class type refer to an instance of a subclass of that type

\zzz{Inheritance and Virtual Methods}
  \1 Java classes can inherit from one class only
    \2 inheritance: \java{class ChildClass extends ParentClass}
    \2 child class gets access to public fields/methods (of course) and protected fields/methods
    \2 child class does not get access to private fields/methods
    \2 TODO abstract classes/methods
  \1 interfaces: \java{class MultiPurpose implements Interface1, IFace2...}
    \2 basically an end around lack of multiple inheritance
    \2 interfaces cannot be instantiated, but you can have a reference of interface type. In that case, the object it points to is a real concrete class, but all that you know about it is that it implements the specified interface
  \1 virtual dispatch:
    \2 all public non-static class methods are virtual.
      This means that if a subclass overrides a parent class method, the decision of which method implementation to use is made at runtime, depending on which type of object the reference actually refers to.
    \2 all interface methods are by definition virtual
    \2 private methods are not virtual, static methods are not virtual
  \1 notable interfaces:
    \2 \java{Iterable<T>}: contains \java{Iterator<T> iterator()} method, to iterate over container
    \2 \java{Iterator<E>}: encapsulates an iteration over a container. Unlike C++ iterators, this iterator knows when it's reached the end, instead of relying on comparison to a one-past-end iterator
      \3 \java{boolean hasNext()}: ask if we're at the end
      \3 \java{E next()}: retrieve next element, and advance iterator
    \2 \java{Runnable}: single \java{void run()} method. This is the interface that threads use

\zzz{Generics}
  \1 \textbf{Subtype Operator}: \texttt{<:}
    \2 if \java{S implements T} then \texttt{S <: T}
    \2 if \texttt{S <: T} then you can freely use an object of type \texttt{S} where type \texttt{T} was required, and it will be type-safe (\texttt{S} can be safely used in that context instead of \texttt{T})
    \2\begin{tabular}{l l l}
    \java{<T extends Number>} & $\rightarrow$ & \texttt{T <: Number} \\
    \java{<T super Number>}   & $\rightarrow$ & \texttt{T :> Number} \\
    \java{T t = new S();}     & $\rightarrow$ & \texttt{S <: T} \\
    \end{tabular}
  \1 type bound: \java{class SortedList<T extends Comparable & Serializable>}
    \2 you use \java{extends} for constraints that are classes or interfaces
    \2 you can bound with \java{extends} or \java{super}
    \2 \java{<T extends Type>}: \java{Type} is an inclusive upper bound on \java{T}
    \2 \java{<T super Type>}: \java{Type} is an inclusive lower bound on \java{T}
  \1 wildcards: \java{static void printAll(List<?> lst)} use \java{?} for when you want to accept an object of any type
    \2 you can also specify \java{<? extends ClassOrInterface...>}
  \1 PECS: Producer Extends, Consumer Super
    \2 to generically assign a \java{T} to something, use \java{<? super T>}
    \2 to generically read a \java{T} from something, use \java{<? extends T>}
\0\begin{minted}{java}
public class CollectionsPECS { 
  public static <T> void copy(List<? super T> dest, 
                              List<? extends T> src) {
    for (int i=0; i<src.size(); i++) 
      dest.set(i,src.get(i)); 
  } 
}
\end{minted}
\0\begin{minted}{java}
import java.lang.*;
class GenericWildcards {
  // T is the binding of the generic parameter
  private static class GenericBox<T> {
    // it is optional, but we need it if we want to do things with that type
    public T t;
    public GenericBox(T t) { this.t = t; }
  }
  private static class NumberBox<T extends Number> {
    public T t;
    public NumberBox(T t) { this.t = t; } 
  }
  public static void printBox(GenericBox<?> b) {
    // here we use the ? wildcard with no type binding because we don't need to do things with that type specifically
    System.out.println(b.t);
  }
  // method generic goes before return type
  public static <T> void printWithParameter(GenericBox<T> b) {
    System.out.println(b.t);
  }
  public static void main(String[] args) {
    // this is using raw types, generally considered bad
    GenericBox rawBox = new GenericBox("asdf1"); // compiler warnings
    // this cast is ok because raw types hold java.lang.Object
    Object o1 = rawBox.t;
    // this causes no warnings for same reason as assignment above doesn't
    printBox(rawBox);
    
    // this is just using an unknown type, java says it's fine
    GenericBox<?> unknownBox = new GenericBox<>("asdf2");
    // this is also OK because <?> explicitly makes the generic parameter as java.lang.Object
    Object o2 = unknownBox.t;
    printBox(unknownBox);

    GenericBox<String> stringBox = new GenericBox<>("asdf3");
    // the type parameter above allows Java to infer that this cast is safe
    String s = stringBox.t;
    System.out.println(s);
    // must specify type between class access and method name (works the same for instance methods too)
    GenericWildcards.<String>printWithParameter(stringBox);

    // correct stuff works like expected
    NumberBox<Integer> nb1 = new NumberBox<>(5);
    System.out.println(nb1.t + 1);
    //
    // this will fail to even allow NumberBox<String> because that type doesn't work
    // NumberBox<String> sb1 = new NumberBox<>("asdf");
    // 
    // this will fail because the inferred type of NumberBox<>("asdf") is NumberBox<String>, which isn't allowed
    // NumberBox<?> sb1 = new NumberBox<>("asdf");
  }
}
\end{minted}


\zzz{Threading}
\\\java{import java.util.concurrent.locks.ReentrantLock;}
\\\java{import java.util.concurrent.locks.Condition;}
\\\zzz{\java{ReentrantLock}}
  \1 \java{ReentrantLock}: basically a mutex
  \1 \java{ReentrantLock.lock()}: acquire the lock (blocking)
    \2 does \textbf{not} throw \java{InterruptedException}
  \1 \java{ReentrantLock.unlock()}: release the lock
    \2 does \textbf{not} throw \java{InterruptedException}
    \2 you should always wrap your locking code in a \java{try{}} block (including the call to \java{lock()} itself) and put the call to \java{unlock()} in a \java{finally{}} block.
    \\This way, \java{unlock()} gets called no matter any exception
\zzz{\java{Condition}}
  \1 created from a lock, allows one thread to send a message to another thread
    \2 create form lock instance using \java{lock.newCondition()}
  \1 \java{await()}: release this lock and wait for the condition to be signaled.
    \\ When the signal happens, \java{await()} will automatically re-acquire the lock before returning
    \\ (this means you will still have to unlock manually)
    \2 you can only \java{await()} when you are holding the lock, and when it returns, you still have the lock, so it acts like you never unlocked it
    \2 \textbf{does} throw \java{InterruptedException}
  \1 \java{signal()}: wake up a single thread that is waiting on the condition
    \2 must be holding lock to signal it's condition
    \2 must manually release lock before other thread will return from \java{await()} (because the other thread must also acquire the lock)
    \2 does \textbf{not} throw \java{InterruptedException}
  \1 \java{signalAll()}: similar to \java{signal()} except that every thread is woken up
    \2 still only one thread will be able to use the lock-protected resource at a time, because locks
    \2 does \textbf{not} throw \java{InterruptedException}
\zzz{Threads}
  \1 \java{static void Thread.sleep(long ms)}: sleep for ms
    \2 throws \java{InterruptedException} if the thread was interrupted before time elapsed
  \1 make new thread with \java{new Thread(Runnable r)}
    \2 start that thread with \java{thread.start()}
\zzz{\java{synchronized}}
\begin{minted}{java}
// inside a method
synchronized(some_object /*may be this*/) {
  // synchronized code here
}
// synchronized getInstance method for singleton
public foo synchronized getInstance() {
  if (inst == null) { inst = new Foo(); }
  return foo;
}
\end{minted}
  \1 methods marked \java{synchronized} are implicitly locked to ensure that only one synchronized method is ever running on a given object at a time
  \1 synchronized statement: synchronize on a specific object manually
  \1 works as a good synchronization mechanism as long as the resource doesn't need to be used directly by multiple objects
  \1 does not allow for conditions
\0
\begin{minted}{java}
import java.lang.*;
import java.util.concurrent.locks.ReentrantLock;
import java.util.concurrent.locks.Condition;
public class Main2 {
  public static class Counter {
    public int count = 0;
    public ReentrantLock lock;
    public Condition updated;
    public Counter() {
      this.lock = new ReentrantLock();
      this.updated = lock.newCondition();
    }
  }
  public static class CounterThread implements Runnable {
    private Counter counter;
    public CounterThread(Counter c) {counter = c;}
    @Override
    public void run() {
      while (true) {
        try {
          counter.lock.lock();
          counter.count += 1;
          System.out.println(counter.count);
          counter.updated.signalAll();
        } 
        // lock() does not throw InterruptedException
        // catch (InterruptedException e) {}
        finally {counter.lock.unlock();}
        
        try {
          Thread.sleep(1000);
        } catch (InterruptedException e) {}
      }
    }
  }

  public static class IntervalPrinter implements Runnable {
    private Counter counter;
    private int mod;
    private String message;
    public IntervalPrinter(Counter c, int mod, String msg) {
      counter = c;
      this.mod = mod;
      message = msg;
    }
    @Override
    public void run() {
      while (true) {
        int val = 0;
        try {
          counter.lock.lock();
          counter.updated.await();
          val = counter.count;
        } 
        catch (InterruptedException e) {}
        finally {counter.lock.unlock();}

        if (val % mod == 0) {
          System.out.println(message);
        }
      }
    }
  }

  public static void main(String []args) {
    Counter c = new Counter();
    new Thread(new IntervalPrinter(c,3,"fizz")).start();
    new Thread(new IntervalPrinter(c,5,"buzz")).start();
    new Thread(new CounterThread(c)).start();
  }
}
\end{minted}


\zzz{{\large Reflection}}
  \1 \java{instanceof} operator: check if an object is an instance of a class or a subclass
\begin{minted}{java}
if (obj instanceof String) {
  // cast is safe because we checked and obj
  String s = (String) obj;
}
\end{minted}

\zzz{\java{java.lang.Class<T>}}
  \1 allows you to reflect on class \java{T}
  \1 to get:
    \2 \java{Class<?> c = SomeClassName.class;}
    \2 \java{Class<?> c = someObjectInstance.getClass();}
    \2 \java{Class<?> c = Class.forName("SomeClassName");}
      \3 throws \java{ClassNotFoundException}
  \1 \java{toString()} returns class declaration (more or less)
  \1 \java{getSimpleName()} returns just the name part of it
    \2 \java{Main.class.getSimpleName()} $\rightarrow$ \java{"Main"}
  \1 \java{Class<? super T> getSuperclass()}
  \1 \java{Class<?>[] getInterfaces()}
    \2 on class object: get interfaces implemented by this class
    \2 on interface object: get interfaces extended by this interface
  \1 methods matching \java{getDeclaredXXX()}
    \2 can see just things declared in the class itself, not from super classes
    \2 can see anything whether public/private/protected/etc...
  \1 methods matching \java{getXXX()}
    \2 operate on whatever the class looks like from an outside observer: only public fields/methods, and from class or super-class/interface
  \1 \java{Method[] getMethods()}
    \2 all public methods, including those inherited from super-classes and implemented in interfaces
  \1 \java{Method getMethod(String name, Class<?>... pt)}
    \2 looks for fields in superclasses, then superinterfaces too
    \2 throws \java{NoSuchMethodException} if method not found
    \2 public methods only
  \1 \java{Method[] getDeclaredMethods()}
    \2 excludes inherited methods; includes any that are declared in class regardless of public, private, static, etc...
  \1 \java{Field[] getFields()}
    \2 returns public fields only!
  \1 \java{Field getField(String name)}
    \2 looks for fields in superinetrfaces, then superclasses too
    \2 throws \java{NoSuchFieldException} if field not found
    \2 only finds public fields
  \1 \java{Constructor<T> getConstructor(Class<?>... pt)}
    \2 get a constructor for \java{T} that matches parameter types \java{Class<?>... pt}
    \2 throws \java{NoSuchMethodException} if there is no constructor matching those parameter types
  \1 \java{T newInstance()}
    \2 create a new instance of \java{T} using the default constructor
  \1 TODO use \java{Constructor} class to create class using non-default constructor

\zzz{\java{java.lang.reflect.Method}}
  \1 \java{String toString()} $\rightarrow$ method prototype as string
    \2 includes modifiers, method name, parameters, etc...
  \1 \java{String getName()} $\rightarrow$ name of method as string
  \1 \java{int getModifiers()} $\rightarrow$ \java{int} representing modifiers
    \2 use \java{java.lang.reflect.Modifier} static methods to check:
    \\ \java{Modifier.isStatic(m.getModifiers())}
  \1 \java{Class<?>[] getParameterTypes()} $\rightarrow$ types of parameters of method
    \2 if no parameters, returns empty array
    \2 does not include implicit \java{this} parameter for instance methods
  \1 \java{Type[] getGenericParameterTypes()}: same, but returns a \java{Type} instance that accurately represents the generic info from the actual source
  \1 \java{Class<?> getReturnType()}: get the return type
    \2 if it's void, it returns a void type
  \1 \java{Object invoke(Object obj, Object... args)}
    \2 invoke a method on an object. Subject to virtual method lookup
    \2 if the method is static, \java{obj} may be null
    \2 if the method returns a primitive type, it is wrapped; if void, returns null
    \2 throws \java{IllegalAccessException} if you can't run that method because it's private or something
    \2 if target method throws, it throws \java{InvocationTargetException} wrapping whatever was thrown

\zzz{\java{java.lang.reflect.Field}}
  \1 \java{TYPE getTYPE(Object obj)}: bunch of methods for getting fields of primitive types
    \2 throws \java{IllegalArgumentException} if type can't be converted (widening conversions only are allowed)
    \2 throws \java{IllegalArgumentException} also if \java{obj} isn't of the right type
  \1 \java{Object get(Object obj)}: get an object field
    \2 if the field is a primitive type, it is wrapped and then returned
    \2 only throws \java{IllegalArgumentException} if \java{obj} isn't of the right type
  \1 \java{String getName()}
  \1 also various \java{setTYPE(Object obj, TYPE value)} and \java{set(Object obj, Object value)} equivalent to get methods


% \begin{minted}{java}
% \end{minted}

\end{outline}
\end{multicols*}
\end{document}

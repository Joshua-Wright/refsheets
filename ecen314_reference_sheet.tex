% Josh_Wright_Resume.tex
% (c) Copyright 2015 Josh Wright
\documentclass[12pt]{article}
\usepackage{verbatim}
% \usepackage{syntonly}
\usepackage{ragged2e}
\usepackage{geometry}
\usepackage{enumitem} % for longenum
\usepackage{setspace}
\usepackage{hyperref}
\usepackage{tabularx}
\usepackage{outlines} % for outline
\usepackage{paralist} % for compactitem (compact itemize)
\usepackage{multicol} % for multicolumn layout
\geometry{letterpaper, margin=0.4in, top=0.3in}
% \geometry{letterpaper, margin=0.5in, top=0.35in, left=1.5in}
\pagenumbering{gobble}
\begin{document}
% \linespread{0.5}
% \begin{center}
% Signals and Systems Reference Sheet
% \hfill \textcopyright \space Josh Wright 2015 \hfill
% Last Updated: \today
% \end{center}

%%%%%%%%%%%%%%%%%%
%% main section %%
%%%%%%%%%%%%%%%%%%
\begin{multicols*}{2}
\begin{flushleft}
\newlist{longenum}{itemize}{5}
\setlist[longenum,1]{nosep,leftmargin=0.4cm,labelwidth=0px,align=left,label=$\bullet$}
\setlist[longenum,2]{nosep,leftmargin=0.4cm,labelwidth=0px,align=left,label=$\ast$}
\setlist[longenum,3]{nosep,leftmargin=0.4cm,labelwidth=0px,align=left,label=-}
\setlist[longenum,4]{nosep,leftmargin=0.4cm,labelwidth=0px,align=left,label=>}
\setlist[longenum,5]{nosep,leftmargin=0.4cm,labelwidth=0px,align=left,label=@}
% \begin{outline}[compactitem]
\begin{outline}[longenum]

%%%%%%%%%%%%%%%%%%%%
%% spacing config %%
%%%%%%%%%%%%%%%%%%%%
% just in case I need even more space
\newlength{\upspacelength}
\setlength{\upspacelength}{0px}
\newcommand{\upspace}{\vspace{\upspacelength}}
% section titles
\newcommand{\zzz}[1]{\upspace \0 \textbf{#1} }
% \newcommand{\zzz}[1]{\0 \hspace{-1.25in} \textbf{#1} \vspace{-10px} }
% makes second-level itemize bullets instead of dashes
% \renewcommand\labelitemii{\labelitemi}
% redefine the sub-headings to inject our space-saver
\let\oldOne\1\let\oldTwo\2\let\oldThree\3\let\oldFour\4
\renewcommand{\1}{\upspace \oldOne   \hspace{-7px}}
\renewcommand{\2}{\upspace \oldTwo   \hspace{-7px}}
\renewcommand{\3}{\upspace \oldThree \hspace{-7px}}
\renewcommand{\4}{\upspace \oldFour  \hspace{-7px}}
\small

ECEN314 refsheet \textcopyright \space Josh Wright \today

\zzz{Metric Prefixes} \\
\begin{tabular}{|c c l l|}                                   \hline
peta  & P     & $10^{ 15}$ & \hfill 1 000 000 000 000 000 \\ \hline
tera  & T     & $10^{ 12}$ & \hfill     1 000 000 000 000 \\ \hline
giga  & G     & $10^{  9}$ & \hfill         1 000 000 000 \\ \hline
mega  & M     & $10^{  6}$ & \hfill             1 000 000 \\ \hline
kilo  & k     & $10^{  3}$ & \hfill                 1 000 \\ \hline
hecto & h     & $10^{  2}$ & \hfill                   100 \\ \hline
deca  & da    & $10^{  1}$ & \hfill                    10 \\ \hline
one   &       & $10^{ 0 }$ & \hfill       1 \hfill \hfill \\ \hline
deci  & d     & $10^{- 1}$ & 0.1                          \\ \hline
centi & c     & $10^{- 2}$ & 0.01                         \\ \hline
milli & m     & $10^{- 3}$ & 0.001                        \\ \hline
micro & $\mu$ & $10^{- 6}$ & 0.000 001                    \\ \hline
nano  & n     & $10^{- 9}$ & 0.000 000 001                \\ \hline
pico  & p     & $10^{-12}$ & 0.000 000 000 001            \\ \hline
femto & f     & $10^{-15}$ & 0.000 000 000 000 001        \\ \hline
\end{tabular}


\zzz{Complex Numbers}
  \1 $z = x+iy = re^{i\theta} = r[\cos(\theta)+i\sin(\theta)]$
  \1 $[r(\cos(\theta)+i\sin(\theta))]^n = r^n[\cos(n\theta)+i\sin(n\theta)]$
  \1 $z^n = (re^{i\theta}) = r^ne^{in\theta}$
  \1 $\frac{1}{i}=-i$
  \1 $\sqrt[n]{z} = \sqrt[n]{r}e^{\frac{\theta}{n}+\frac{2k\pi}{n}}$ for $n\in N^*$ (ints $\geq0$)
  \1 $e^{j\theta} = \cos(\theta) + j\sin(\theta)$
  \1 $e^{-j\theta} = \cos(\theta) - j\sin(\theta)$
  \1 $\cos(\theta) = \frac{1}{2}(e^{j\theta} + e^{-j\theta})$
  \1 $\sin(\theta) = \frac{1}{2j}(e^{j\theta} - e^{-j\theta})$
  \1 normalized: $sinc(t) = \frac{\sin(\pi t)}{\pi t}$

\zzz{Trig}
  \1 $\cos^2(a)+\sin^2(a)=1$
  \1 $\cos(2a)=\cos^2(a) - \sin^2(a) = 2\cos^2(a)-1 = 1-2\sin^2(a)$
  \1 $\sin(2a) = 2\sin(a)\cos(a)$
  \1 $\cos^2(a) = \frac{1}{2}(1 + \cos(2a))$
  \1 $\sin^2(a) = \frac{1}{2}(1 - \cos(2a))$


\zzz{Signals}
  \1 \textbf{Even/Odd}
  \\ even: $x(-t) =  x(t)$ for all $t$
  \\ odd:  $x(-t) = -x(t)$ for all $t$
  \1 \textbf{Auto Correlation:} compare signal with a time-delayed version of itself
  \\ $\phi(\tau) = \int_{-\infty}^{\infty}x(t)*x(t+\tau)dt$
    \2 peaks will be at multiples of the period
  \1 \textbf{Cross Correlation:} like autocorrelation, but for two different signals
  \\ $\phi(\tau) = \int_{-\infty}^{\infty}x_1(t)*x_2(t+\tau)dt$
    \2 to easily tell if one signal is a shifted version of another
  \1 Shifting and scaling: just always remember you're replacing \textbf{just} $t$ with an expression involving $t$
  \1 \textbf{Unit Step} Signal
    \2 $u(t) = \{_{1, n>0}^{0, n<0}$
  \1 \textbf{Discrete Unit Impulse} Signal
  \\ $\delta[n] = \{_{1, n=0}^{0, n\not=0}$
    \2 any discrete signal can be represented as a sum of shifted unit impulse signals
    \2 $\delta[n] = u[n] - u[n-1]$
  \1 \textbf{Continuous Unit Impulse} Signal
  \\ $x(t) = \delta(t) = \{_{\infty, t=0}^{0, t\not=0}$
    \2 discontinuous at $t=0$
    \2 $\int_{-\infty}^{\infty} \delta(t) dt = 1$
    \2 pick out values from discrete function: (shifting property)
    \\ $\int_{-\infty}^{\infty} \delta(t)*f(t) dt = f(0)$
    \\ $\int_{-\infty}^{\infty} \delta(t-a)*f(t) dt = f(a)$
  \1 \textbf{Shifting Property:} $\int_{-\infty}^{\infty}x(t)\sigma(t-t_0)dt = x(t_0)$
  \1 \textbf{Bounded:} $x(t)\leq M$ for all $t$, some $M$
    \2 unbounded signals typically are infinite at some time instant
  \1 \textbf{Causal} iff $x(t)=0$ for all $t<0$
  \1 \textbf{Energy:} 
    $E_x = \int_{-\infty}^{\infty}|x(t)|^2 dt$
    \2 signal is an energy signal if $0 < E_x < \infty$
  \1 \textbf{Power:} 
    $P_x = \frac{1}{T}\int_{T} |x(t)|^2 dt$
    \2 (for periodic signals)
    \2 signal is an power signal if $0 < P_x < \infty$

\zzz{Convolution}
  \1 $\sum_{k=-\infty}^{\infty} x(k) h(n-k)$ 
    or $\int_{-\infty}^{\infty} x(\tau) h(t-\tau) d\tau$
  \1 graphically:
    \2 choose one function to be $h$
    \2 flip around origin with $t \rightarrow -t$
    \2 shift back and forth on form $h(t-\tau)$
      \2 shift is reversed because the negative
    \2 multiply by $x(t)$ and then sum
  \1 if the system is LTI invariant, then the convolution of $x(t)$ with the impulse response $h(t)$ is the same as if $x(t)$ were the input of the system
  \1 convolution with shifted unit impulse is the same as shifting the original system: $h(t)*\sigma(t-a) = h(t-a)$
  \1 Step response is just convolution with impulse response.
    worked out: $u(t)*h(t)=\int_{-\infty}^{t}h(\tau) d\tau$
    \2 only works for LTI systems!

\zzz{Geometric Series}
  \1 $\sum_{k=0}^{\infty} ar^n = \frac{a}{1-r}$
  \1 $a$ is first term of the series
    \\ $r$ is ratio between terms: $r = \frac{a_1}{a_0} = \frac{a_2}{a_1} \ldots$

\zzz{Systems}
  \1 A system is an operation that transforms an input signal into an output signal
    \2 you can add/subtract signals
    \2 composing signals (one input to another) is convolution
      \\ (easier to just shift if input is shifted unit step (because LTI))
  \1 \textbf{BIBO stability:} output is stable iff input signal is stable
    \2 also if impulse response $\int_{-\infty}^{\infty} |h(t)| dt < \infty$ (for LTI systems)
    \2 bounded: $h(t)<M$ for all $t$ and some $M$
  \1 \textbf{Memory:} iff the system depends on past or future values of the input
  \1 \textbf{Causality:} iff the output depends only on the current or past values of the input
    \2 (cannot depend on future values of input)
  \1 \textbf{Invertibility:} iff the system's input can be recovered from the output
  \1 \textbf{Time Invariance:} iff shifting the input signal shifts the output
    \2 integral is time invariant
  \1 \textbf{Superposition:} additive commutativity
    \2 $H\{x_1(t)+x_2(t)\} = H\{x_1(t)\}+H\{x_2(t)\}$
  \1 \textbf{Homogeneity:}
    \2 $H\{a x(t)\} = a H\{x(t)\}$
  \1 \textbf{Linearity:} iff satisfies Superposition and Homogeneity
    \2 $H\{a x_1(t)+b x_2(t)\}$ $= a H\{x_1(t)\}+b H\{x_2(t)\}$
    \2 averaging filter is linear
  \1 \textbf{LTI:} both Linear and Time Invariant
    \2 simplest systems
  \1 system from block diagram:
    \2 add/subtract signals just like you would
    \2 for signals $h_1(t) \rightarrow h_2(t)$ (in series), you get $y(t) = h_1(t)*h_2(t)$ (convolution of the two signals)
    \2 basic method is to keep combining adjacent signal blocks using convolution, scaling, and addition until you get a single block
  \1 system from differential equation:
    \2 solve equation for $y(t)$
    \2 stuff in terms of input goes on the left; output on the right
    \2 add constants scaling to each output, and sum it all together

\zzz{Linearity}
  \1 system is linear if it satisfies superposition (additive) and homogeneity (scalable)
    \2 superposition: $h(a) + h(b) = h(a + b)$
    \2 homogeneity: $a h(b) = h(a b)$

\zzz{Noise}
  \1 unwanted signals generated externally or internally
  \1 thermal noise is a thing

\zzz{Impulse Response}
  \1 output of a system when the input is $\sigma(t)$
  \1 \begin{tabular}{l l}
    memoryless if  & $h(t) = c\sigma(t)$                          \\
    causal if      & $h(t)=0$ for $t<0$                           \\
    BIBO stable if & $\int_{-\infty}^{\infty} |h(t)| dt < \infty$ \\
    invertible if  & $h(t)*h^{inv}(t)=\sigma(t)$                  \\
  \end{tabular}
    \2 same for discrete time

% ----------------- post exam 1

\zzz{even/odd signals}
  \1 $f(t) = f_e(t) + f_o(t)$
  \1 $f_e(t) = \frac{1}{2} ( f(t) + f(-t) )$
  \1 $f_o(t) = \frac{1}{2} ( f(t) - f(-t) )$

\zzz{Fourier Series}
  \1 Harmonic: $e^{jk2\pi F_0 t}$
  \1 Synthesis: $f(t) = \sum_{k=-\infty}^{\infty} X[k] e^{jk2\pi F_0 t}$
  \1 Analysis: $X[k] = \frac{1}{T_p}\int_{0}^{T_p} x(t) e^{-jk2\pi F_0 t} dt$
    \2 note the different sign!
  \1 $X[k] = C_k$
  \1 Parseval's theorem: (energy of a signal)
    $\int_{-\infty}^{\infty} |x(t)|^2 dt = \frac{1}{2\pi} \int_{-\infty}^{\infty} |X(j\omega)|^2 d\omega$
  \1 FT of fraction of two polynomials: use partial fraction decomposition

\zzz{Fourier Transform}
  \1 $x(t) =       \int_{-\infty}^{\infty} X(j\omega) e^{ j \omega t} d\omega$
  \1 $X(j\omega) = \int_{-\infty}^{\infty} x(t)       e^{-j \omega t} dt     $

\zzz{Fourier Properties}%todo make a table of these
  \1 linearity: $z(t) = ax(t)+by(t) \leftrightarrow Z(k) = aX(k)+bY(k) $
  \1 time shift: $x(t-t_0) \leftrightarrow X(k)e^{-jk\omega_0t_0}$
  \1 frequency shift: $x(t)e^{jk_0\omega_0t} \leftrightarrow X(k-k_0)$
  \1 time scaling: same coefficients, $x(at) \rightarrow \omega = a\omega_0$ (for $a>0$)
  \1 time reversal: $x(-t) \leftrightarrow X(-k)$
  \1 convolution: $x(t)\ast z(t) \leftrightarrow TX(k)Z(k) $
  \1 multiplication: $x(t)z(t) \leftrightarrow \sum_{l=-\infty}^{\infty} X(k)Z(k-l)$
    \2 similar to convolution
  \1 derivative: $\frac{d}{dt}(x(t)) \leftrightarrow jk\omega_0 X(k) $
  \1 integral: $\int_{-\infty}^{t} x(t) dt \leftrightarrow \frac{1}{jk\omega_0} X(k) $
  \1 Symmetry: if $x(t)=x_r(t)+jx_i(t)$ then $x^*(t)=x_r(t)-jx_i(t)$
  \1 if $x(t)$ is real and even, $X(k)$ is real and even
  \1 if $x(t)$ is real and odd, $X(k)$ is imaginary and odd
  \1 unnormalized sinc: $sinc(t) = \frac{sin(t)}{t}$
  \1 normalized sinc: $sinc(t) = \frac{sin(\pi t)}{\pi t}$

\zzz{Frequency Response}%L17
  \1 how the system will respond to a particular frequency
  \1 the Fourier Transform of the impulse response
    \\ (we don't have to convolve it here, just multiply since it's frequency domain)
  \1 find using $H(\omega)=\frac{Y(\omega)}{X(\omega)}$
    \2 if the starting equation is expressed as a differential equation, you can (usually) derive this from that.
  \1 usually represented as $H(\omega)=|H(\omega)|e^{j\theta_H(\omega)}$ (magnitude and phase)
    \2 magnitude: $|H(\omega)|$, phase: $\theta_H(\omega)$
  \1 magnitude and phase can be linearly combined

\zzz{Sampling}
  \1 sampling: independent variable (input); continuous $\rightarrow$ discrete
  \1 quantization: dependent variable (output); continuous $\rightarrow$ discrete
  \1 aliasing: different signals being indistinguishable after sampling due to sampling rate.
  \1 $F_{CT}$: continuous time frequency; $f_{DT}$: discrete time frequency, $F_s$: sampling frequency 
  \1 $F_{CT} \rightarrow f_{DT}$ is a many to one mapping
    \2 Folding Frequency = $F_s/2$, where frequency wraps-around
    \2 restrict to one-to-one: satisfy $|F| < F_s / 2$
  \1 \textbf{Nyquist Rate:} signal with maximum frequency $f_{max}$ can be recovered exactly if it is sampled at least $f_s > 2f_{max}$
  \1 continuous to discrete:
    \2 $x[n] = x_a(n/F_s)$
    \2 $X(f) = F_s \sum_{k=-\infty}{\infty} X_a[(f-k)F_s]$
  \1 sampling is equivalent to convolving with a delta chain (dirac comb) % I think, TODO
  \1 under-sampling: $f_s<2 f_{max}$. Generally bad

% http://blogs.mathworks.com/steve/2010/01/18/relationship-between-continuous-time-and-discrete-time-fourier-transforms/

\zzz{DFT}
  \1 discrete in both time domain and frequency domain
  \1 zero padding (on the right) increases frequency domain resolution
  \1 frequency domain is on range $[0, 2\pi)$
  \1 TODO
\zzz{DTFT}
  \1 Discrete Time Fourier Transform
  \1 Discrete in time, continuous in frequency
  \1 $X(e^{j\Omega}) = \sum_{n=-\infty}^{\infty} x[n] e^{-j\Omega n}$
  \1 $x[n] = \frac{1}{2\pi} \int_{-\pi}^{\pi} X(e^{j\Omega}) e^{j\Omega n} d\Omega$
  \1 frequency domain is \textbf{periodic}, range $[-\pi,\pi]$ is repeated
    \2 if aliasing happens ($F_s < 2 f_{max}$) then parts will overlap, and won't work right
  \1 maximum frequency $\rightarrow \pi$; sampling frequency $\rightarrow 2\pi$

%%%%%%%%%%%%%%%%%
%% final stuff %%
%%%%%%%%%%%%%%%%%


\zzz{Laplace Transform}
  \1 $X(s) = \int_{0}^{\infty} x(t) e^{st} dt$
  \1 $s = \sigma - j\omega$
  \1 relation to Fourier Transform: $X(j\omega)=X(s)|_{s=j\omega, \sigma=0}$
  \1 zeros: numerator is 0, so value is 0
  \1 poles: denominator is 0, so value is $+$ or $-$ $\infty$
    \2 effect on impulse response on $t>0$:
    \\ poles left of $s=0$: decaying exponential
    \\ poles right of $s=0$: increasing exponential
    \\ (reversed on $t<0$)
  \1 Specified on a Region of Convergence (ROC)
    \1 ROC must contain no poles
  \1 Unilateral: integral starts from 0
    \2 assumes signals is 0 before $t=0$ 
    \2 i.e. multiplied by $u(t)$
  \1 Bilateral: starts from $-\infty$
    \2 same as unilateral iff $x(t)=0$ for $t<0$
  \1 Shifting:
    \2 $x(t-T) \leftrightarrow e^{-sT}X(s)$
    \2 $T$ such that $x(t-T)u(t) = x(t-T)u(t-T)$
    \\ i.e. doesn't shift any non-zero part of the signal left of $t=0$
    \\ (Doesn't apply for Bilateral transform)
  \1 Initial Value Theorem:
    \2 $\lim_{s\to\infty}sX(s)=\lim_{t\to0^+}x(t)=x(0^+)$
    \2 order of numerator $<$ denominator
    \2 $x(t)=0$ for $t<0$
  \1 Final Value Theorem:
    \2 $\lim_{s\to0}sX(s)=\lim_{t\to\infty}x(t)=x(\infty)$
    \2 $x(t)=0$ for $t<0$, $x(t)$ finite as $t\to\infty$
    \2 all poles on left of plane with at most one at $s=0$
  \1 System Response:
    \2 take Laplace Transform to get equation in form $a(s)Y(s) - c(s) = b(s)X(s)$
    \2 solve $Y(s)=\frac{b(s)X(s)}{a(s)} + \frac{c(s)}{a(s)} = Y^{(f)}(s) + Y^{(n)}(s)$
    \2 $Y^{(f)}(s)$: forced response; $Y^{(n)}(s)$: natural response
    \2 Frequency Response (transfer function): $H(\omega)=\frac{Y(\omega)}{X(\omega)}$
  \1 Stability: system is stable $\leftrightarrow$ all poles left of $s=0$

\zzz{Filters}
  \1 multiply signal by a filter to filter it
  \1 pass: allow through (not filtered out)
  \1 stop: filter out, remove
  \1 $something$ pass filter passes $something$; same for stop
  \1 low pass filter: passes $\omega$ lower than $b$, drops higher
    $H(\omega)=\{ _{1:|\omega|<b}^{0:|\omega|>b}$
  \1 high pass filter: opposite of low pass filter
  \1 band pass filter: pass a specific band of frequency.
    \2 That frequency is specified by magnitude, so it can be on the positive or negative side of the graph
  \1 Notch filter: band stop filter with a narrow stop band
  \1 An ideal filter has exact edges, but real filters don't
    \2 This is impossible in practice. 
      Typically there is vertical variation inside the pass band and stop band, and also a trans band ($\omega_s$), as transition between pass and stop band.
      % more details later in this course apparently
  \1 series RC circuit with $x(t)$ as input and $y(t)$ on capacitor: 
    $RC y'(t) + y(t) = x(t)$
    or $y'(t) + \frac{1}{RC}y(t) = \frac{1}{RC}x(t)$
    \2 capacitor: $I(t) = C*\frac{d}{dt}(V(t))$
    \2 inductor: $V(t) = L*\frac{d}{dt}(I(t))$
  \1 highpass and band-stop filters do not have bandwidth
  \1 Absolute Bandwidth: width of passband
  \1 3-dB Bandwidth: (for low pass): width of passband where $H(j\omega) \geq |H(0)|/\sqrt{2}$
    \2 because low pass has max at $\omega=0$. Different for other filter types
    \2 AKA half power bandwidth
  \1 Bode Plot: frequency response ($H(s=jw, \sigma=0)$) in $dB$ as a function of logarithm of frequency
  \1 ideal filter has linear phase response
    \2 so that the response to different frequencies is the same
    \2 constant delay is linear phase response (it's just $H(j\omega)=e^{-j\omega a}$)
  \1 ideal filter magnitude response is 1 in passband and 0 in stopband
  \1 real filter has $E, \delta$ and transition band $(\omega_p, \omega_s)$ such that:
    \2 passband: $1 - E \leq H(j\omega) < 1$ for $0\leq |\omega| \leq \omega_p$
    \2 stopband: $|H(j\omega)| < \delta$ for $\omega > \omega_s$
  \1 FIR filter: Finite Impulse Response
    \2 finite memory (lower startup transient time)
    \2 always BIBO stable
    \2 desired magnitude response with exactly linear phase response
  \1 IIR filter: Infinite Impulse Response
    \2 output governed by recursive linear constant coefficient differential equations
    \2 uses z transform
    \2 allows shorter (lower order?) filters
    \2 has phase distortion (nonlinear phase response)
    \2 non-finite transient startup
  \1 Butterworth Filter
    \2 looks like a nice simple slope, relatively large transition band though
    \2 $\omega_c$: cutoff frequency
    \2 $|H(j\omega)|^2 = 1/(1+(\frac{\omega}{\omega_c})^{1/(2K)}), K=1,2,3,\ldots$
    \\ $H(s)H(-s)|_{s=j\omega}=|H(jw)|^2$
    \2 passband: $\omega_p = \omega_c(\frac{\epsilon}{1-\epsilon})^{1/(2K)}$
    \2 stopband: $\omega_s = \omega_c(\frac{1-\delta}{\delta})^{1/(2K)}$
    \2 poles: $\omega_c e^{i\pi (2k+1)/(2K)}, k=0,1,\ldots,(2K-1)$
      \3 never pure imaginary
      \3 left of $s=0$ belong to $H(s)$, right of $s=0$ belong to $H(-s)$
    \2 to get transfer function, find poles and make fraction from there
  \1 Chebyshev Filter
    \2 poles on ellipse in $s$-plane
    \2 ripples in either passband or stopband, not both
    \2 smaller transition band
  \1 Elliptic Filter
    \2 smallest transition band
    \2 ripples in both passband and stopband

\zzz{Communication Systems}
  \1 frequency range: what frequencies are best transmitted, and what are lost
  \1 modulation: embedding a signal inside a carrier signal that propagates better
  \1 simultaneous transmission (multiplexing): use modulation to transmit more than one signal simultaneously

\zzz{Amplitude Modulation (AM)}
  \1 mix (convolve) payload and carrier in frequency domain (multiply in time domain)
  \1 carrier is typically a high frequency cosine, which is two impulses in frequency domain, so the convolution makes two shifted copies of the input signals, each centered around $\pm$ carrier frequency
  \1 demodulation: recover original signal: shift it back to the origin and apply low-pass filter
  \1 Frequency Modulation is another (different) kind of modulation



\end{outline}
\end{flushleft}
\end{multicols*}
\end{document}
